\documentclass[12pt, spanish]{article}
\usepackage[none]{hyphenat}
\usepackage[utf8]{inputenc}
\usepackage[T1]{fontenc}
\usepackage[spanish]{babel}

\usepackage{geometry}
\def \margin {20mm}
\geometry{
	a4paper,
    left=\margin,
	right=\margin,
	top=\margin,
	bottom=\margin
}

\usepackage{hyperref}

\usepackage{enumitem}
% \setenumerate{label=(\alph*),leftmargin=0.6cm}
% \setitemize{label=---,leftmargin=0.6cm}

\usepackage{titlesec}
\titleformat{\section}{\huge}{\thesection}{.3em}{}
\titleformat{\subsection}{\Large}{}{0pt}{}
\titleformat{\subsubsection}{\large}{}{0pt}{}
\renewcommand{\thesection}{\Roman{section}}

\usepackage{amssymb}
\usepackage{mathtools}

% Theorems
\usepackage{amsthm}
\usepackage{thmtools}


\newtheorem*{defin}{Definición}

\def\theaxsection{}
\newcommand{\setaxsection}[1]{\def\theaxsection{#1}\setcounter{ax}{0}}

\newtheoremstyle{axstyle}{\topsep}{\topsep}{\itshape}{0pt}{\bfseries}{.}{ }
    {\thmname{#1} \theaxsection\thmnumber{#2}\textnormal{\thmnote{ (#3)}}}
\theoremstyle{axstyle}
\newtheorem{ax}{Axioma}
\newtheoremstyle{axbstyle}{\topsep}{\topsep}{\itshape}{0pt}{\bfseries}{.}{ }
    {\thmname{#1}\thmnote{ #3}}
\theoremstyle{axbstyle}
\newtheorem{axb}{Axioma}

% No indentation
\setlength\parindent{0pt}

\let\emptyset\varnothing

\renewcommand{\geq}{\geqslant}
\renewcommand{\leq}{\leqslant}

\DeclareMathSymbol{*}{\mathbin}{symbols}{"01}

\begin{document}
\fontfamily{cmss}\selectfont

{ \Large\textbf{Axiomas de Hilbert}\hfill Adrián Lattes} \noindent\rule{17cm}{1pt}
\vspace*{1em}


En este documento resumiremos los axiomas y principales definiciones de la
axiomatización de Hilbert de la geometría euclídea, considerando el caso
restringido de la geometría plana.

\section{Nociones primitivas}

El sistema axiomático de Hilbert parte de considerar cinco nociones primitivas:
dos términos primitivos (puntos y líneas) y cuatro relaciones primitivas (orden,
pertenencia, congruencia entre segmentos y congruencia entre ángulos).

\begin{itemize}
    \item Denotaremos los \textit{puntos} con letras mayúsculas $A,B,C,\dots$
    \item Denotaremos las \textit{líneas} con letras minúsculas $a,b,c,\dots$
    \item La relación de \textit{incidencia} es una relación binaria entre
        puntos y rectas. Dado un punto $A$ y una recta $l$ escribiremos $A\sim
        l$ para denotar que $A$ y $l$ están relacionados mediante la relación de
        incidencia.
    \item La relación de \textit{orden} es una relacion ternaria entre puntos.
        Para tres puntos $A, B, C$ escribiremos $A * B * C$ para indicar que 
        están relacionados mediante la relación de orden.
    \item La relación de \textit{congruencia entre segmentos} es una relación
        binaria entre \textit{segmentos} (noción definida más adelante).
        Denotaremos esta relacion con el símbolo $\cong$.
    \item La relación de \textit{congruencia entre ángulos} es una relación
        binaria entre \textit{ángulos} (noción definida más adelante). Para
        denotar esta relación usaremos el mismo símbolo que para la congruencia
        entre segmentos, $\cong$. Por el contexto quedará claro a que relación
        nos referimos.
\end{itemize}

\section{Axiomas}

\subsection{Axiomas de incidencia}
\setaxsection{I}

\begin{ax}\label{I1}
    Para cada par de puntos distintos $A, B$ existe una única recta que los
    contiene. Denotraremos esta recta por $AB$.
\end{ax}

\begin{ax}\label{I2}
    Cada línea contiene al menos dos puntos distintos.
\end{ax}

\begin{ax}\label{I3}
   Existen tres puntos no colineares. Es decir, existen $A$, $B$ y $C$ tales que
   $AB\neq BC$.
\end{ax}

\begin{defin}
  Decimos que tres puntos distintos son \textbf{colineares} si existe una recta 
  que los contiene.
\end{defin}

\subsection{Axiomas de orden}
\setaxsection{B}

\begin{ax}\label{B1}
  Si un punto $B$ está entre $A$ y $C$ ($A * B * C$) entonces $A, B, C$ son
  distintos, están en una misma línea y $C * B * A$.
\end{ax}

\begin{ax}\label{B2}
  Para cada dos puntos distintos $A,B$ existe un punto $C$ tal que $A * B * C$.
\end{ax}

\begin{ax}\label{B3}
  Dados tres puntos distintos en una línea, uno y sólo uno de ellos está entre
  los otros dos.
\end{ax}

\begin{ax}[Pasch]\label{B4} 
  Sean $A, B, C$ tres puntos no colineares y $l$ una línea que no contenga a 
  ninguno de estos puntos. Si $l$ contiene un punto $D$ que está entre $A$ y $B$ 
  ($A * D * B$) entonces también debe contener un punto entre $A$ y $C$ o un
  punto entre $B$ y $C$.
\end{ax}

\begin{defin}
  Dados dos puntos distintos $A, B$ definimos el \textbf{segmento}
  $\overline{AB}$ como el conjunto de puntos que contiene a $A, B$ y a todos los
  puntos que están entre ellos. Diremos que $A$ y $B$ son los extremos del
  segmento $\overline{AB}$.
\end{defin}

\begin{defin}
  Diremos que dos puntos \textbf{están del mismo lado del plano} respecto de una
  recta si el segmento que los une no contiene ningún punto de la recta.

  Diremos que dos puntos están del mismo \textbf{lado de una recta} respecto de
  un punto si el segmento que los une no contiene a dicho punto.
\end{defin}

\begin{defin}
  Dados dos puntos distintos $A, B$, definimos el \textbf{rayo}
  $\overrightarrow{AB}$ como el conjunto que contiene a $A$ y a todos los puntos
  de la línea $AB$ tales que están del mismo lado de $A$ que $B$. Dado un rayo
  $\overrightarrow{AB}$ llamaremos \textbf{vértice} del rayo al punto $A$.
\end{defin}

\begin{defin}
  Un \textbf{ángulo} es la unión de dos rayos $\overrightarrow{AB}$ y
  $\overrightarrow{AC}$ con el mismo vértice y no contenidos en una misma recta.
  Denotaremos dicho ángulo por $\angle ABC$.
\end{defin}

\subsection{Axiomas de congruencia}
\subsubsection{Para segmentos}
\setaxsection{C}

\begin{ax}\label{C1}
  Dados un segmento $\overline{AB}$ y un rayo $r$ con vértice $C$, existe un
  único punto $D$ en el rayo $r$ tal que $\overline{AB}\cong\overline{CD}$.
\end{ax}

\begin{ax}\label{C2}
  Si $\overline{AB}\cong\overline{CD}$ y $\overline{AB}\cong\overline{EF}$
  entonces $\overline{CD}\cong\overline{EF}$. Además cada segmento es congruente
  con sí mismo.
\end{ax}

\begin{ax}[Suma]\label{C3}
  Dados tres puntos $A, B, C$ en una línea y tales que $A * B * C$ y otros tres
  puntos $D, E, F$ en una línea tales que $D * E * F$, si
  $\overline{AB}\cong\overline{DE}$ y $\overline{BC}\cong\overline{EF}$ entonces
  $\overline{AC}\cong\overline{DF}$.
\end{ax}

% \begin{defin}
%   Dados dos segmentos $\overline{AB}$, $\overline{CD}$ y un orden $A,B$ entre
%   los extremos del segmento $\overline{AB}$
% \end{defin}


\subsubsection{Para ángulos}

\begin{ax}\label{C4}
  Dados un ángulo $\angle ABC$ y un rayo $\overrightarrow{DF}$, fijado un lado
  del plano de la línea $DF$, existe un único rayo $\overrightarrow{DE}$ en
  dicho lado tal que $\angle BAC\cong\angle EDF$.

\end{ax}

\begin{ax}\label{C5}
  Para cada tres ángulos $\alpha, \beta, \gamma$, si $\alpha\cong\beta$ y
  $\alpha\cong\gamma$ entonces $\beta\cong\gamma$. Además cada ángulo es
  congruente con sí mismo.
\end{ax}

\begin{ax}[SAS]\label{C6}
  Sean tres triángulos $ABC$ y $DEF$ tales que
  $\overline{AB}\cong\overline{DE}$, $\overline{AC}\cong\overline{DF}$ y $\angle
  BAC\cong\angle EDF$. Entonces los dos triángulos son congruentes, es decir
  $\overline{BC}\cong\overline{EF}$, $\angle ABC\cong\angle DEF$ y $\angle
  ACB\cong\angle DFE$.
\end{ax}

\subsection{Planos de Hilbert}

\begin{defin}
    Un \textbf{plano de Hilbert} es un modelo de la teoría formada por las
    anteriores nociones primitivas y axiomas de incidencia, orden y congruencia.
\end{defin}

En un plano de Hilbert se pueden recuperar bastantes de los resultados de la
geometría plana de los elementos de Euclides, pero no todos. 

Por ejemplo no se puede demostrar la primera proposición del libro primero, en
la que se enuncia la construcción de un triángulo equilátero dado uno de sus
lados. Para esto será necesario introducir otro axioma.

\section{Otros axiomas}

\begin{defin}
  Dados dos puntos distintos $O$ y $A$ se define el \textbf{círculo} $\Gamma$ de centro
  $O$ y radio $\overline{OA}$ como el conjunto de los puntos $B$ tales que
  $\overline{OA}\cong\overline{OB}$. El punto $O$ se llama \textbf{centro} del 
  círculo y el segmento $\overline{OA}$ es un \textbf{radio} del círculo.

  Se dice que un punto $C$ está en el \textbf{interior} de un círculo $\Gamma$
  si $B=O$ o $\overline{OB}<\overline{OA}$. Si $\overline{OA}<\overline{OC}$ el
  punto $C$ es \textbf{exterior} a $\Gamma$.
\end{defin}

\begin{axb}[E]\label{E}
  Dados dos círculos $\Gamma, \Delta$, si $\Delta$ contiene al menos un punto en
  el interior de $\Gamma$ y otro en el exterior, entonces $\Gamma$ y $\Delta$ se
  intersecan.
\end{axb}

\begin{axb}[P]\label{P}
  Para cada punto $P$ y línea $l$ existe como mucho una línea paralela a $P$ y
  que pase por $P$.
\end{axb}

\end{document}
