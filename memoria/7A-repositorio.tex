\section{Repositorio y organización del código}\label{sec:repositorio}

El código desarrollado en este trabajo se encuentra en un \textit{repositorio
	git}, alojado en el siguiente enlace:
\url{https://github.com/haztecaso/euclidean-geometry-lean}.

En este repositorio se encuentran todas las demostraciones que se han
desarrollado pero no han sido incluidas en este documento debido a su extensión.

La estructura de los ficheros del repositorio es la siguiente:

\dirtree{%
	.0 .
	.1 memoria.
	.2 memoria.pdf.
	.2 presentacion.pdf.
	.1 src.
	.2 basic_defs.lean.
	.2 incidence_geometry.
	.3 basic.lean.
	.3 propositions.lean.
	.2 order_geometry.
	.3 basic.lean.
	.3 propositions.lean.
	.2 congruence_geometry.
	.3 basic.lean.
	.2 parallels_independence.lean.
	.2 examples.lean.
	.1 readme.md.
}

En el fichero \texttt{readme.md}, que se puede leer cómodamente en el enlace del
repositorio, se pueden consultar las instrucciones de ejecución del código,
junto a una breve descripción del contenido de cada fichero.

