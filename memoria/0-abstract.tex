\begin{abstract}
	Este trabajo explora el uso del asistente de demostración Lean, un
	lenguaje que implementa una teoría de tipos, el
	\textit{Cálculo de construcciones inductivas}, utilizando la axiomática de
	Hilbert de la geometría euclídea como guía motivadora para el
	aprendizaje.

	Comentamos aspectos generales de la formalización de matemáticas asistida
	por computadoras, sus aplicaciones y ventajas; se introducen conceptos
	básicos del funcionamiento de Lean; se analiza el trabajo de
	formalización realizado; y se concluye proponiendo una línea de continuación
	del trabajo.
\end{abstract}

\begin{abstract}[Abstract]
	This paper explores the use of the Lean proof assistant, a language that
	implements a type theory, the \textit{Calculus of Inductive Constructions},
	using Hilbert's axiomatics of Euclidean geometry as a motivating guide for
	learning.

	We discuss the general aspects of the computer-assisted formalization of
	mathematics, its applications and advantages; we also introduce basic
	concepts of Lean operation; we analyze the formalization work carried out;
	and we conclude by proposing a line of continuation of the work.

\end{abstract}

\newpage









