\documentclass[12pt, spanish]{TFG}

\usepackage{lipsum}

\title{Formalización de las matemáticas con Lean.\\ Un caso de estudio: Geometría euclídea plana.}
\date{Septiembre de 2023}
\author{Adrián Lattes Grassi}
\subtitle{Facultad de Ciencias Matemáticas\\Trabajo dirigido por Jorge Carmona Ruber}

\begin{document}

% PORTADA
\maketitle
\emptypage

% RESUMEN - ABSTRACT

\begin{abstract}

  Este trabajo explora el uso del asistente de demostración \textit{Lean}, un
  lenguaje de programación que implementa una teoría de tipos útil para
  verificar formalmente demostraciones matemáticas, para formalizar enunciados 
  y resultados de la axiomática de Hilbert de la geometría euclídea plana. Esta
  teoría servirá de guía para introducir el uso del asistente y exponer cómo
  puede ser utilizado para construir relaciones de equivalencia, modelos de
  una teoría y demostrar la independencia entre axiomas.

\end{abstract}

\begin{abstract}[Abstract]

 Resumen traducido al inglés.

\end{abstract}

\newpage

% ÍNDICE
\tableofcontents
\newpage

% CONTENIDO

\addcontentsline{toc}{section}{Introducción}
\section*{Introducción}

Los asistentes de demostración nos permiten formalizar definiciones, enunciados
de proposiciones y teoremas, demostraciones, y verficar estas definiciones.
Formalizar matemáticas consiste en digitalizar enunciados y resultados
escribiéndolos en un lenguaje de programación que garantiza, mediante una
correspondencia entre una teoríá de tipos y la lógica, la validez de cada paso.

Algunos beneficios de formalizar enunciados y resultados matemáticos mediante un
asistente de demostración son:

\begin{itemize} 

  \item El proceso de formalización requiere explicitar todos los detalles de
      las demostraciones. \redactar{Garantiza la comprensión.}

  \item 

\end{itemize}

\subsection{Contenidos del trabajo}

\begin{itemize}

    \item Introducción a la formalización de enunciados y demostraciones
        mediante el uso de tácticas

    \item Tratamiento de distintos patrones de demostración comunes en la
        geometría euclídea
        \begin{itemize}
            \item ¿Qué es un modelo?
            \item ¿Cómo podemos trabajar con relaciones de equivalencia?
            \item ¿Cómo demostramos independencia de axiomas?
        \end{itemize}

    \item Desarrollo y demostración de resultados. Se formalizan enunciados y
        demostraciones de las primeras secciones de axiomas. Posteriormente se
        plantea cómo se podría demostrar la independencia del axioma de las
        paralelas y se formalizan los enunciados necesarios dejando pendientes
        ciertas construcciones.

\end{itemize}

\newpage
\section{Lean}

En esta sección se introducen los elementos básicos del lenguaje para


\newpage
\section{La geometría de Hilbert}

En esta sección se introduce la teoría de la geometría de
Hilbert~\cite{hilbert1902grundlagenfoundations}.
limitaremos al caso de la geometría plana. Introducción de los problemas de la
formalización de Euclides en los Elementos.

Se utilizan nociones y construcciones intuitivas, basadas en justificaciones
mediante dibujos, pero no incluidas dentro de la axiomática.


\newpage
\section{Formalizando la geometría de Hilbert en Lean}

\newpage
\addcontentsline{toc}{section}{Referencias}
\bibliography{referencias}

\end{document}
