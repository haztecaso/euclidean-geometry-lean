\section{Introducción a Lean}

\begin{description}
	\item[Tipos y conjuntos] Analogías y diferencias
	\item[hipótesis] Los resultados pueden depender de unas hipótesis,
		enunciadas antes de los \lstinline{:}.
	\item[El tipo Prop]
	\item[Implicación y funciones]
	\item[Doble implicación]
	\item[and y or como tipos inductivos]
	\item[Términos true y false]
	\item[La negación]
\end{description}

% \subsection{Tácticas básicas}

% \begin{description}[font=\sffamily\bfseries, leftmargin=1.5cm, style=nextline]
% 	\item[refl] Sirve para demostrar metas de la forma \lstinline{A = A}. Por
% 		ejemplo, podemos demostrar que todo punto es igual a si mismo:

% 		\begin{lstlisting}
% lemma point_refl (A : Point) : A = A :=
% begin
%   refl
% end
%         \end{lstlisting}
% 	\item[rw] Para reescribir
% 	\item[exact] Si la meta es  \lstinline{⊢ X} y \lstinline{x} es un término de
% 		tipo \lstinline{X} se puede usar la táctica \lstinline{exact x} para
% 		cerrar la meta.
% 	\item[apply] Demostrar hacia atrás
% 	\item[intro] Introducir hipótesis
% 		\begin{lstlisting}
% lemma equal_comm (A B : Point) : A = B → B = A :=
% begin
%   intro h,
%   sorry
% end
%         \end{lstlisting}


% 	\item[split] Dividir la meta
% 	\item[left, right] Modificar la meta
% 	\item[exfalso] Cambia la meta a false
% 	\item[by_cases] Para dividir la demostración en casos.
% 	\item[have] Demostrar resultados intermedios
% 	\item[cases] Varios usos. Descomponer and, or o exists.

% \end{description}

% \subsection{Tácticas de nivel alto}

% \begin{description}
% 	\item[simp] Para demostrar igualdades
% \end{description}

\section{Introducción a Lean}
\subsection{Teoría de tipos informal}

Lo primero que se necesita en matemáticas para poder formalizar afirmaciones es
un lenguaje formal con el cual expresarlas. Normalmente se utiliza la lógica de
primer orden, en el contexto de la teoría de conjuntos. Lean, sin embargo,
utiliza un sistema deductivo diferente, el de la teoría de tipos.

En lógica de primer se tiene una aserción fundamental, \textit{que una
	proposición dada tenga una prueba}. Es decir, cada proposición $P$ da lugar a la
aserción correspondiente \textit{P tiene una prueba}. Mediante ciertas reglas de
transformación, y a veces una serie de axiomas, se pueden construir nuevas
pruebas. Por ejemplo, dada la regla de inferencia \textit{de $A$ se deduce $A$ o
	$B$} y la aserción \textit{A tiene una prueba}, se obtiene la aserción \textit{A
	o B tiene una prueba}.

En teoría de tipos la aserción fundamental es \textit{que un término tenga un
	tipo}. Dados un término $a$ y un tipo $A$, si \textit{$a$ tiene tipo $A$}
escribimos $a:A$. Esta es misma notación es la utilizada por Lean, en el que por
ejemplo podemos expresar la aserción \textit{3 es un número natural} con el
código \lstinline{3 : ℕ}.

Es importante no confundir esta notación con la de una
relación interna a nuestro lenguaje. Mientras que en teoría de conjuntos
utilizamos la relación de pertenencia $\in$ para expresar que un elemento
primitivo (un conjunto) está contenido en otro, en teoría de tipos no podemos
considerar los términos o los tipos por separado. La noción fundamental es la
pertenencia de tipos y cada término tiene que estar siempre acompañado por su
tipo. En teoría de tipos además existen otras aserciones, como la de igualdad
entre términos de un mismo tipo. Dados $a:A$ y $b:A$, se tiene la aserción
\textit{$a$ y $b$ son dos términos de tipo A iguales por definición}, y
escribimos $a\equiv b : A$.

La teoría de tipos también puede utilizarse para expresar afirmaciones y
demostraciones matemáticas. Las afirmaciones se codifican mediante los tipos y
las demostraciones mediante construcciones de términos de un tipo dado. Es
decir, se puede interpretar la aserción $a : A$ como \textit{$a$ es una
	demostración de $A$}. Esta interpretación da lugar a una analogía entre la
lógica proposicional y la teoría de tipos, llamada correspondencia de
Curry-Howard. Sin entrar en detalles, a cada proposición lógica se le puede
asignar un tipo, y a cada demostración de un enunciado un término del tipo
correspondiente al enunciado.

\todoref{HTT}

Existen distintas elecciones de reglas de transformación que considerar en
teoría de tipos, que dan lugares a distintas versiones de la teoría de tipos.
Lean implementa una una versión de la teoría de tipos dependiente conocida como
el \textit{Calculus of Constructions}.

La base de muchas teorías de tipos es el lambda cálculo, un modelo universal de
computación introducido por \textit{Alonzo Church} en los años treinta. Sin
entrar en su formalización, en el lambda cálculo se consideran dos operaciones
fundamentales para tratar con funciones, la abstracción y la evaluación.

\begin{itemize}

	\item \textbf{Abstracción}. Es el mecanismo de definición de funciones mediante
	      la introducción de parámetros. Dado un término \lstinline{x + 1 : ℕ},
	      mediante la sintaxis \lstinline{λ x : ℕ, (x + 1 : ℕ)} se convierte la
	      variable libre \lstinline{x} en una variable ligada por la abstracción, a la
	      que llamamos parámetro de la función.

	      Es importante recordar que cada término tiene que ir acompañado del tipo al
	      que pertenece. En esta expresión estamos indicando que tanto el parámetro
	      \lstinline{x} como el resultado de la función, \lstinline{x+1}, son de tipo
	      \lstinline{ℕ}. Es decir, tenemos una función que dado un número natural
	      devuelve otro número natural. Esto también puede escribirse como
	      \lstinline{(λ x, x + 1) : ℕ → ℕ}.

	      Lean además incluye notación para definir funciones que devuelven otras
	      funciones, lo cual es muy útil para tratar funciones que reciben más de un
	      parámetro (\todoref{Ver currificacion}). Las siguientes líneas de código
	      definen expresiones equivalentes
	      \begin{lstlisting}
λ a : α, λ b : β, a
λ (a : α) (b : β), a
\end{lstlisting}
	      que representan el mismo término, de tipo \lstinline{α → β → α}.

	\item \textbf{Evaluación}. Consiste en aplicar funciones, pasándoles los
	      valores de los argumentos que evaluar. Por ejemplo la expresión
	      \lstinline{(λ x : ℕ, (x + 1) : ℕ) (1 : ℕ)} indica que estamos evaluando la
	      función \lstinline{(λ x, x + 1) : ℕ → ℕ} con el parámetro \lstinline{x}
	      sustituido por el término \lstinline{1 : ℕ} (para que la sustitución pueda
	      realizarse los tipos deben coincidir). Mediante un proceso de
	      reducción se obtiene el término \lstinline{2 : ℕ}.

\end{itemize}

\todoref{Theorem proving in Lean}

Como se ve en los ejemplos, el código fuente de Lean se pueden incluir
caracteres unicode, como \lstinline{λ}, \lstinline{→} o \lstinline{ℕ}. Esta
característica del lenguaje es muy útil a la hora de escribir código lo más
cercano posible a las notaciones a las que estamos acostumbrados a usar en
matemáticas. La inclusión de estos caracteres está facilitada en el entorno de
desarrollo, escribiendo comandos que empiecen por \texttt{\textbackslash} estos
se reemplazarán por el caracter correspondiente. Por ejemplo al escribir
\texttt{\textbackslash to} este comando se reemplazará automáticamente por el
caracter \lstinline{→}, \texttt{\textbackslash lambda} por \lstinline{λ} y
\texttt{\textbackslash N} por \lstinline{ℕ}.

\todo{Incluir en algún lugar explicación sobre entorno de desarrollo (plugin vscode) y sus características (¿Apéndice?).}

Que cada término sea siempre considerado junto a su tipo no significa que sea
siempre necesario explicitar dicho tipo. Lean tiene un mecanismo llamado
\textit{inferencia de tipos} que le permite deducir automáticamente el tipo de
un término cuando no ha sido explicitado pero el contexto aporta información
suficiente.
Por ejemplo, cuando definimos la función \lstinline{λ x : ℕ, (x + 1 : ℕ)} no es
necesario incluir la segunda anotación de tipo. Dada la expresión
\lstinline{x + 1} y sabiendo por contexto que \lstinline{x : ℕ} el sistema de
inferencia de tipos deduce que la suma de dos números naturales también es un
número natural, por lo que se infiere el tipo \lstinline{ℕ}.

\subsection{Proposiciones en Lean}

En Lean se tiene el tipo \lstinline{Prop} para expresar las proposiciones
mediante la analogía de \textit{proposiciones como tipos} dada por la
correspondencia de Curry-Howard.

La correspondencia de Curry-Howard afirma que estas funciones de la
teoría de tipos se comportan de la misma forma que la implicación en lógica. Por
tanto en Lean utilizaremos el símbolo \lstinline{→} para referirnos tanto a
funciones como a implicaciones dentro de una proposición. Dadas dos
proposiciones \lstinline{p q : Prop} podemos construir la proposición
\lstinline{p → q : Prop}, que se interpreta como \lstinline{p} \textit{implica}
\lstinline{q}.

\subsection{Introducción de símbolos}

En lean existen distintas formas de introducir nuevos símbolos en el entorno
actual.

\subsubsection{Constantes}%

Mediante los comandos \lstinline{constant} y \lstinline{constants} se
pueden introducir símbolos en el entorno, postulando su existencia. Este comando
equivale por tanto a considerar nuevos axiomas sobre la existencia de los
símbolos que introduce.
\begin{lstlisting}
constant a : ℕ
constants (b : ℤ) (c : ℂ)
\end{lstlisting}


\subsubsection{Definiciones}%
Si no queremos introducir nuevos axiomas podemos definir nuevos símbolos
mediante el comando \lstinline{def}. Como estamos definiendo un símbolo, es
necesario proporcionar el tipo y término que queremos asignar al símbolo:
\begin{lstlisting}
def succ : ℕ → ℕ := λ n, n + 1
def succ' (n : ℕ) : ℕ := n + 1 -- Otra forma de definir succ
\end{lstlisting}

Para introducir parámetros de funciones se puede omitir la notación
\lstinline{λ}, incluyendo las variables parametrizadas antes de los dos puntos
que anotan el tipo de la definición.

Esta notación de introducción de parámetros es muy útil y simple, pero puede
resultar demasiado explícita. Veamos por ejemplo cómo se puede definir la
función identidad, que dado un término de un tipo devuelve el mismo término.
Si queremos que la identidad definida sea sea general y se le puedan aplicar términos
de cualquier tipo necesitaremos introducir el tipo del término como un parámetro
adicional.

\begin{lstlisting}
def id1 (α : Type*) (e : α) := e
\end{lstlisting}

El problema de esta definición es que cada vez que se quiera utilizar será
necesario proporcionarle como primer argumento el tipo del término que se le
quiere pasar, por ejemplo \lstlisting{id1 ℕ 0}.
Pero la función identidad que queremos considerar recibe un solo argumento.
Como a cada término acompaña siempre su tipo, dado el término \lstinline{e : α},
el sistema de inferencia de tipos de Lean es capaz de deducir automáticamente el
tipo \lstinline{α}. Solo falta indicar en la definición cuál es el parámetro
cuya identificación queremos delegar al sistema de inferencia de tipos.

\begin{lstlisting}
def id2 {α : Type*} (e : α) := e
\end{lstlisting}

Así los parámetros indicados entre llaves, llamados \textit{parámetros
	implícitos}, serán deducidos automaticamente.


\newpage

