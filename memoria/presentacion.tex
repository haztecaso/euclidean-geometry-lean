\documentclass[t, aspectratio=169]{beamer}
\usepackage[utf8]{inputenc}
\usepackage[T1]{fontenc}
\usepackage[spanish]{babel}

\renewcommand{\labelitemi}{$\bullet$}
\renewcommand{\labelitemii}{$\bullet$}
\renewcommand{\labelitemiii}{$\bullet$}

\usepackage{caption}
\usepackage{csquotes}
\usepackage{graphicx}

\usepackage[none]{hyphenat}
\usepackage{lmodern}
\usepackage{microtype}

\newcommand{\todo}[1]{\noindent\textcolor{red}{\textbf{Pendiente: }#1}}
\newcommand{\todoref}[1]{\noindent\textcolor{cyan}{\textbf{Referencia: }#1}}
\newcommand{\redactar}[1]{\noindent\textcolor{blue}{\bfseries #1}}

% _ escaping
\catcode`\_=12

% Lean3 formatting
\usepackage{amssymb}
\usepackage{amsthm}
\usepackage{upgreek}
\usepackage{listings}
\usepackage{xcolor}
\definecolor{backcolor}{rgb}{0.9, 0.9, 0.9}   % grey
\definecolor{keywordcolor}{rgb}{0.7, 0.1, 0.1}   % red
\definecolor{commentcolor}{rgb}{0.4, 0.4, 0.4}   % grey
\definecolor{symbolcolor}{rgb}{0.0, 0.1, 0.6}    % blue
\definecolor{sortcolor}{rgb}{0.1, 0.5, 0.1}      % green
\definecolor{errorcolor}{rgb}{1, 0, 0}           % bright red
\definecolor{stringcolor}{rgb}{0.5, 0.3, 0.2}    % brown
\definecolor{tacticcolor}{rgb}{0.0, 0.1, 0.6}

\renewcommand{\lstlistingname}{Listado}

\def\lstlanguagefiles{lstlean.tex}

\lstdefinestyle{leanBase}{
	language=lean,
	aboveskip=1em,
	belowskip=.5em,
}

\lstdefinestyle{leanSimple}{
	style=leanBase,
	frame=trlb,
	framesep=10pt,
	rulecolor=\color{gray},
	basicstyle=\fontsize{10}{11}\selectfont\ttfamily,
	xleftmargin=-.75cm,
	xrightmargin=-.75cm,
	aboveskip=1em,
	belowskip=.5em,
	% framexleftmargin=1em,
	% framexrightmargin=1em,
}
\lstdefinestyle{leanFull}{
	style=leanSimple,
	numbers=left,
	% numbersep=5pt,
	xleftmargin=0cm,
	framexleftmargin=.75cm,
	numberstyle=\scriptsize\ttfamily\color{gray},
	% belowskip=-.5em,
	% backgroundcolor=\color{backcolor}, % background color
}

\lstset{style=leanSimple}

\newcommand{\lstleanfull}[3]{
	\lstinputlisting[
		style=leanFull,
		firstnumber=#2,
		firstline=#2,
		lastline=#3,
		title={\ttfamily \footnotesize src/#1}
	]{../src/#1}
}

\newtheorem{prop}{Proposición}
\newtheorem{tma}{Teorema}
\newtheoremstyle{defin}{14.0pt plus 2.0pt minus 4.0pt}{0mm}{}{}{\bfseries}{.}{.5em}{}
\theoremstyle{defin}
\newtheorem*{defin*}{Definición}

\def\theaxsection{}
\newcommand{\setaxsection}[1]{\def\theaxsection{#1}\setcounter{ax}{0}}

\newtheoremstyle{axstyle}{\topsep}{\topsep}{\itshape}{0pt}{\bfseries}{.}{ }
{\thmname{#1} \theaxsection\thmnumber{#2}\textnormal{\thmnote{ (#3)}}}
\theoremstyle{axstyle}
\newtheorem{ax}{Axioma}
\newtheoremstyle{axbstyle}{\topsep}{\topsep}{\itshape}{0pt}{\bfseries}{.}{ }
{\thmname{#1}\thmnote{ #3}}
\theoremstyle{axbstyle}
\newtheorem{axb}{Axioma}





\setbeamertemplate{navigation symbols}{}
\setbeamertemplate{frametitle}[default][center]
\addtobeamertemplate{frametitle}{\let\insertframetitle\insertsectionhead}{}
\addtobeamertemplate{frametitle}{\let\insertframesubtitle\insertsubsectionhead}{}

\makeatletter
\CheckCommand*\beamer@checkframetitle{\@ifnextchar\bgroup\beamer@inlineframetitle{}}
\renewcommand*\beamer@checkframetitle{\global\let\beamer@frametitle\relax\@ifnextchar\bgroup\beamer@inlineframetitle{}}
\makeatother

\usepackage{graphicx}
\usepackage[export]{adjustbox}
\usepackage{caption}
\captionsetup[figure]{font=small}
\usepackage{float}

\usepackage{multicol}

\title{Formalización de las matemáticas con Lean.\\ Un caso de estudio: Geometría euclídea plana.}
\author{Adrián Lattes Grassi}
\date{18 de septiembre de 2023}
\subtitle{Facultad de Ciencias Matemáticas.\\Trabajo dirigido por Jorge Carmona Ruber.}

\lstdefinestyle{leanBeamer}{
  style=leanBase,
  frame=,
  basicstyle=\fontsize{8}{9}\selectfont\ttfamily,
  escapeinside=||
}
\lstset{style=leanBeamer}

\setlength\columnsep{50pt}

\begin{document}
\frame{\titlepage}

\begin{frame}[fragile]
	\begin{multicols}{2}
		\begin{lstlisting}
lemma neq_lines_have_at_most_one_common_point 
  (Point : Type*) {Line : Type*} 
  [ig : incidence_geometry Point Line] :
  ∀ l m : Line, l ≠ m → 
    (∃! A : Point, is_common_point A l m) 
    ∨ ¬ have_common_point Point l m := 
\end{lstlisting}
		\columnbreak
		\hfill
	\end{multicols}
\end{frame}










\begin{frame}[fragile]
	\begin{multicols}{2}
		\begin{lstlisting}
lemma neq_lines_have_at_most_one_common_point 
  (Point : Type*) {Line : Type*} 
  [ig : incidence_geometry Point Line] :
  ∀ l m : Line, l ≠ m → 
    (∃! A : Point, is_common_point A l m) 
    ∨ ¬ have_common_point Point l m := 
begin
\end{lstlisting}
		\columnbreak
		\begin{lstlisting}

		\end{lstlisting}
	\end{multicols}
\end{frame}










\begin{frame}[fragile]
	\begin{multicols}{2}
		\begin{lstlisting}
lemma neq_lines_have_at_most_one_common_point 
  (Point : Type*) {Line : Type*} 
  [ig : incidence_geometry Point Line] :
  ∀ l m : Line, l ≠ m → 
    (∃! A : Point, is_common_point A l m) 
    ∨ ¬ have_common_point Point l m := 
begin
\end{lstlisting}
		\columnbreak
		\begin{lstlisting}
Point Line: Type u
ig: incidence_geometry Point Line






⊢ ∀ (l m : Line), l ≠ m → 
  (∃! A : Point, is_common_point A l m) 
  ∨ ¬ have_common_point Point l m
		\end{lstlisting}
	\end{multicols}
\end{frame}










\begin{frame}[fragile]
	\begin{multicols}{2}
		\begin{lstlisting}
lemma neq_lines_have_at_most_one_common_point 
  (Point : Type*) {Line : Type*} 
  [ig : incidence_geometry Point Line] :
  ∀ l m : Line, l ≠ m → 
    (∃! A : Point, is_common_point A l m) 
    ∨ ¬ have_common_point Point l m := 
begin
  intros l m
\end{lstlisting}
		\columnbreak
		\begin{lstlisting}
Point Line: Type u
ig: incidence_geometry Point Line






⊢ ∀ (l m : Line), l ≠ m → 
  (∃! A : Point, is_common_point A l m) 
  ∨ ¬ have_common_point Point l m
		\end{lstlisting}
	\end{multicols}
\end{frame}










\begin{frame}[fragile]
	\begin{multicols}{2}
		\begin{lstlisting}
lemma neq_lines_have_at_most_one_common_point 
  (Point : Type*) {Line : Type*} 
  [ig : incidence_geometry Point Line] :
  ∀ l m : Line, l ≠ m → 
    (∃! A : Point, is_common_point A l m) 
    ∨ ¬ have_common_point Point l m := 
begin
  intros l m,
\end{lstlisting}
		\columnbreak
		\begin{lstlisting}
Point Line: Type u
ig: incidence_geometry Point Line
l m : Line





⊢ l ≠ m → 
  (∃! A : Point, is_common_point A l m) 
  ∨ ¬ have_common_point Point l m
		\end{lstlisting}
	\end{multicols}
\end{frame}










\begin{frame}[fragile]
	\begin{multicols}{2}
		\begin{lstlisting}
lemma neq_lines_have_at_most_one_common_point 
  (Point : Type*) {Line : Type*} 
  [ig : incidence_geometry Point Line] :
  ∀ l m : Line, l ≠ m → 
    (∃! A : Point, is_common_point A l m) 
    ∨ ¬ have_common_point Point l m := 
begin
  intros l m,
  contrapose
\end{lstlisting}
		\columnbreak
		\begin{lstlisting}
Point Line: Type u
ig: incidence_geometry Point Line
l m : Line





⊢ l ≠ m → 
  (∃! A : Point, is_common_point A l m) 
  ∨ ¬ have_common_point Point l m
		\end{lstlisting}
	\end{multicols}
\end{frame}










\begin{frame}[fragile]
	\begin{multicols}{2}
		\begin{lstlisting}
lemma neq_lines_have_at_most_one_common_point 
  (Point : Type*) {Line : Type*} 
  [ig : incidence_geometry Point Line] :
  ∀ l m : Line, l ≠ m → 
    (∃! A : Point, is_common_point A l m) 
    ∨ ¬ have_common_point Point l m := 
begin
  intros l m,
  contrapose,
\end{lstlisting}
		\columnbreak
		\begin{lstlisting}
Point Line: Type u
ig: incidence_geometry Point Line
l m : Line





⊢ ¬((∃! A : Point, is_common_point A l m) 
    ∨ ¬ have_common_point Point l m) → 
  ¬ l ≠ m
		\end{lstlisting}
	\end{multicols}
\end{frame}










\begin{frame}[fragile]
	\begin{multicols}{2}
		\begin{lstlisting}
lemma neq_lines_have_at_most_one_common_point 
  (Point : Type*) {Line : Type*} 
  [ig : incidence_geometry Point Line] :
  ∀ l m : Line, l ≠ m → 
    (∃! A : Point, is_common_point A l m) 
    ∨ ¬ have_common_point Point l m := 
begin
  intros l m,
  contrapose,
  push_neg
\end{lstlisting}
		\columnbreak
		\begin{lstlisting}
Point Line: Type u
ig: incidence_geometry Point Line
l m : Line





⊢ ¬((∃! A : Point, is_common_point A l m) 
    ∨ ¬ have_common_point Point l m) → 
  ¬ l ≠ m
		\end{lstlisting}
	\end{multicols}
\end{frame}










\begin{frame}[fragile]
	\begin{multicols}{2}
		\begin{lstlisting}
lemma neq_lines_have_at_most_one_common_point 
  (Point : Type*) {Line : Type*} 
  [ig : incidence_geometry Point Line] :
  ∀ l m : Line, l ≠ m → 
    (∃! A : Point, is_common_point A l m) 
    ∨ ¬ have_common_point Point l m := 
begin
  intros l m,
  contrapose,
  push_neg,
\end{lstlisting}
		\columnbreak
		\begin{lstlisting}
Point Line: Type u
ig: incidence_geometry Point Line
l m : Line





⊢ (¬∃! A : Point, is_common_point A l m) 
   ∧ have_common_point Point l m → 
  l = m
		\end{lstlisting}
	\end{multicols}
\end{frame}










\begin{frame}[fragile]
	\begin{multicols}{2}
		\begin{lstlisting}
lemma neq_lines_have_at_most_one_common_point 
  (Point : Type*) {Line : Type*} 
  [ig : incidence_geometry Point Line] :
  ∀ l m : Line, l ≠ m → 
    (∃! A : Point, is_common_point A l m) 
    ∨ ¬ have_common_point Point l m := 
begin
  intros l m,
  contrapose,
  push_neg,
  rintro ⟨not_unique, hlm⟩
\end{lstlisting}
		\columnbreak
		\begin{lstlisting}
Point Line: Type u
ig: incidence_geometry Point Line
l m : Line





⊢ (¬∃! A : Point, is_common_point A l m) 
   ∧ have_common_point Point l m → 
  l = m
		\end{lstlisting}
	\end{multicols}
\end{frame}










\begin{frame}[fragile]
	\begin{multicols}{2}
		\begin{lstlisting}
lemma neq_lines_have_at_most_one_common_point 
  (Point : Type*) {Line : Type*} 
  [ig : incidence_geometry Point Line] :
  ∀ l m : Line, l ≠ m → 
    (∃! A : Point, is_common_point A l m) 
    ∨ ¬ have_common_point Point l m := 
begin
  intros l m,
  contrapose,
  push_neg,
  rintro ⟨not_unique, hlm⟩,
\end{lstlisting}
		\columnbreak
		\begin{lstlisting}
Point Line: Type u
ig: incidence_geometry Point Line
l m : Line
not_unique: ¬∃! A : Point, is_common_point A l m
hlm: have_common_point Point l m



⊢ l = m
		\end{lstlisting}
	\end{multicols}
\end{frame}










\begin{frame}[fragile]
	\begin{multicols}{2}
		\begin{lstlisting}
lemma neq_lines_have_at_most_one_common_point 
  (Point : Type*) {Line : Type*} 
  [ig : incidence_geometry Point Line] :
  ∀ l m : Line, l ≠ m → 
    (∃! A : Point, is_common_point A l m) 
    ∨ ¬ have_common_point Point l m := 
begin
  intros l m,
  contrapose,
  push_neg,
  rintro ⟨not_unique, hlm⟩,
  rw exists_unique at not_unique
\end{lstlisting}
		\columnbreak
		\begin{lstlisting}
Point Line: Type u
ig: incidence_geometry Point Line
l m : Line
not_unique: ¬∃! A : Point, is_common_point A l m
hlm: have_common_point Point l m



⊢ l = m
		\end{lstlisting}
	\end{multicols}
\end{frame}










\begin{frame}[fragile]
	\begin{multicols}{2}
		\begin{lstlisting}
lemma neq_lines_have_at_most_one_common_point 
  (Point : Type*) {Line : Type*} 
  [ig : incidence_geometry Point Line] :
  ∀ l m : Line, l ≠ m → 
    (∃! A : Point, is_common_point A l m) 
    ∨ ¬ have_common_point Point l m := 
begin
  intros l m,
  contrapose,
  push_neg,
  rintro ⟨not_unique, hlm⟩,
  rw exists_unique at not_unique,
\end{lstlisting}
		\columnbreak
		\begin{lstlisting}
Point Line: Type u
ig: incidence_geometry Point Line
l m : Line
not_unique: ¬∃ A : Point, 
  is_common_point A l m 
  ∧ ∀ B : Point, is_common_point B l m → B = A
hlm: have_common_point Point l m

⊢ l = m
		\end{lstlisting}
	\end{multicols}
\end{frame}










\begin{frame}[fragile]
	\begin{multicols}{2}
		\begin{lstlisting}
lemma neq_lines_have_at_most_one_common_point 
  (Point : Type*) {Line : Type*} 
  [ig : incidence_geometry Point Line] :
  ∀ l m : Line, l ≠ m → 
    (∃! A : Point, is_common_point A l m) 
    ∨ ¬ have_common_point Point l m := 
begin
  intros l m,
  contrapose,
  push_neg,
  rintro ⟨not_unique, hlm⟩,
  rw exists_unique at not_unique,
  push_neg at not_unique
\end{lstlisting}
		\columnbreak
		\begin{lstlisting}
Point Line: Type u
ig: incidence_geometry Point Line
l m : Line
not_unique: ¬∃ A : Point, 
  is_common_point A l m 
  ∧ ∀ B : Point, is_common_point B l m → B = A
hlm: have_common_point Point l m

⊢ l = m
		\end{lstlisting}
	\end{multicols}
\end{frame}










\begin{frame}[fragile]
	\begin{multicols}{2}
		\begin{lstlisting}
lemma neq_lines_have_at_most_one_common_point 
  (Point : Type*) {Line : Type*} 
  [ig : incidence_geometry Point Line] :
  ∀ l m : Line, l ≠ m → 
    (∃! A : Point, is_common_point A l m) 
    ∨ ¬ have_common_point Point l m := 
begin
  intros l m,
  contrapose,
  push_neg,
  rintro ⟨not_unique, hlm⟩,
  rw exists_unique at not_unique,
  push_neg at not_unique,
\end{lstlisting}
		\columnbreak
		\begin{lstlisting}
Point Line: Type u
ig: incidence_geometry Point Line
l m : Line
not_unique: ∀ A : Point, is_common_point A l m → 
  (∃ B : Point, is_common_point B l m ∧ B ≠ A)
hlm: have_common_point Point l m


⊢ l = m
		\end{lstlisting}
	\end{multicols}
\end{frame}










\begin{frame}[fragile]
	\begin{multicols}{2}
		\begin{lstlisting}
lemma neq_lines_have_at_most_one_common_point 
  (Point : Type*) {Line : Type*} 
  [ig : incidence_geometry Point Line] :
  ∀ l m : Line, l ≠ m → 
    (∃! A : Point, is_common_point A l m) 
    ∨ ¬ have_common_point Point l m := 
begin
  intros l m,
  contrapose,
  push_neg,
  rintro ⟨not_unique, hlm⟩,
  rw exists_unique at not_unique,
  push_neg at not_unique,
  cases hlm with A hA
\end{lstlisting}
		\columnbreak
		\begin{lstlisting}
Point Line: Type u
ig: incidence_geometry Point Line
l m : Line
not_unique: ∀ A : Point, is_common_point A l m → 
  (∃ B : Point, is_common_point B l m ∧ B ≠ A)
hlm: have_common_point Point l m


⊢ l = m
		\end{lstlisting}
	\end{multicols}
\end{frame}










\begin{frame}[fragile]
	\begin{multicols}{2}
		\begin{lstlisting}
lemma neq_lines_have_at_most_one_common_point 
  (Point : Type*) {Line : Type*} 
  [ig : incidence_geometry Point Line] :
  ∀ l m : Line, l ≠ m → 
    (∃! A : Point, is_common_point A l m) 
    ∨ ¬ have_common_point Point l m := 
begin
  intros l m,
  contrapose,
  push_neg,
  rintro ⟨not_unique, hlm⟩,
  rw exists_unique at not_unique,
  push_neg at not_unique,
  cases hlm with A hA,
\end{lstlisting}
		\columnbreak
		\begin{lstlisting}
Point Line: Type u
ig: incidence_geometry Point Line
l m : Line
not_unique: ∀ A : Point, is_common_point A l m → 
  (∃ B : Point, is_common_point B l m ∧ B ≠ A)
A: Point
hA: is_common_point A l m

⊢ l = m
		\end{lstlisting}
	\end{multicols}
\end{frame}










\begin{frame}[fragile]
	\begin{multicols}{2}
		\begin{lstlisting}
lemma neq_lines_have_at_most_one_common_point 
  (Point : Type*) {Line : Type*} 
  [ig : incidence_geometry Point Line] :
  ∀ l m : Line, l ≠ m → 
    (∃! A : Point, is_common_point A l m) 
    ∨ ¬ have_common_point Point l m := 
begin
  intros l m,
  contrapose,
  push_neg,
  rintro ⟨not_unique, hlm⟩,
  rw exists_unique at not_unique,
  push_neg at not_unique,
  cases hlm with A hA,
  rcases not_unique A hA with ⟨B, ⟨hB, hAB⟩⟩
\end{lstlisting}
		\columnbreak
		\begin{lstlisting}
Point Line: Type u
ig: incidence_geometry Point Line
l m : Line
not_unique: ∀ A : Point, is_common_point A l m → 
  (∃ B : Point, is_common_point B l m ∧ B ≠ A)
A: Point
hA: is_common_point A l m

⊢ l = m
		\end{lstlisting}
	\end{multicols}
\end{frame}










\begin{frame}[fragile]
	\begin{multicols}{2}
		\begin{lstlisting}
lemma neq_lines_have_at_most_one_common_point 
  (Point : Type*) {Line : Type*} 
  [ig : incidence_geometry Point Line] :
  ∀ l m : Line, l ≠ m → 
    (∃! A : Point, is_common_point A l m) 
    ∨ ¬ have_common_point Point l m := 
begin
  intros l m,
  contrapose,
  push_neg,
  rintro ⟨not_unique, hlm⟩,
  rw exists_unique at not_unique,
  push_neg at not_unique,
  cases hlm with A hA,
  rcases not_unique A hA with ⟨B, ⟨hB, hAB⟩⟩,
\end{lstlisting}
		\columnbreak
		\begin{lstlisting}
Point Line: Type u
ig: incidence_geometry Point Line
l m : Line
A B: Point
hA: is_common_point A l m
hB: is_common_point B l m
hAB: B ≠ A

⊢ l = m
\end{lstlisting}
	\end{multicols}
\end{frame}










\begin{frame}[fragile]
	\begin{multicols}{2}
		\begin{lstlisting}
lemma neq_lines_have_at_most_one_common_point 
  (Point : Type*) {Line : Type*} 
  [ig : incidence_geometry Point Line] :
  ∀ l m : Line, l ≠ m → 
    (∃! A : Point, is_common_point A l m) 
    ∨ ¬ have_common_point Point l m := 
begin
  intros l m,
  contrapose,
  push_neg,
  rintro ⟨not_unique, hlm⟩,
  rw exists_unique at not_unique,
  push_neg at not_unique,
  cases hlm with A hA,
  rcases not_unique A hA with ⟨B, ⟨hB, hAB⟩⟩,
  rw ne_comm at hAB
\end{lstlisting}
		\columnbreak
		\begin{lstlisting}
Point Line: Type u
ig: incidence_geometry Point Line
l m : Line
A B: Point
hA: is_common_point A l m
hB: is_common_point B l m
hAB: B ≠ A

⊢ l = m
\end{lstlisting}
	\end{multicols}
\end{frame}










\begin{frame}[fragile]
	\begin{multicols}{2}
		\begin{lstlisting}
lemma neq_lines_have_at_most_one_common_point 
  (Point : Type*) {Line : Type*} 
  [ig : incidence_geometry Point Line] :
  ∀ l m : Line, l ≠ m → 
    (∃! A : Point, is_common_point A l m) 
    ∨ ¬ have_common_point Point l m := 
begin
  intros l m,
  contrapose,
  push_neg,
  rintro ⟨not_unique, hlm⟩,
  rw exists_unique at not_unique,
  push_neg at not_unique,
  cases hlm with A hA,
  rcases not_unique A hA with ⟨B, ⟨hB, hAB⟩⟩,
  rw ne_comm at hAB,
\end{lstlisting}
		\columnbreak
		\begin{lstlisting}
Point Line: Type u
ig: incidence_geometry Point Line
l m : Line
A B: Point
hA: is_common_point A l m
hB: is_common_point B l m
hAB: A ≠ B

⊢ l = m
\end{lstlisting}
	\end{multicols}
\end{frame}










\begin{frame}[fragile]
	\begin{multicols}{2}
		\begin{lstlisting}
lemma neq_lines_have_at_most_one_common_point 
  (Point : Type*) {Line : Type*} 
  [ig : incidence_geometry Point Line] :
  ∀ l m : Line, l ≠ m → 
    (∃! A : Point, is_common_point A l m) 
    ∨ ¬ have_common_point Point l m := 
begin
  intros l m,
  contrapose,
  push_neg,
  rintro ⟨not_unique, hlm⟩,
  rw exists_unique at not_unique,
  push_neg at not_unique,
  cases hlm with A hA,
  rcases not_unique A hA with ⟨B, ⟨hB, hAB⟩⟩,
  rw ne_comm at hAB,
  exact unique_of_exists_unique (ig.I1 hAB) ⟨hA.1, hB.1⟩ ⟨hA.2, hB.2⟩
\end{lstlisting}
		\columnbreak
		\begin{lstlisting}
Point Line: Type u
ig: incidence_geometry Point Line
l m : Line
A B: Point
hA: is_common_point A l m
hB: is_common_point B l m
hAB: A ≠ B

⊢ l = m
\end{lstlisting}
	\end{multicols}
\end{frame}










\begin{frame}[fragile]
	\begin{multicols}{2}
		\begin{lstlisting}
lemma neq_lines_have_at_most_one_common_point 
  (Point : Type*) {Line : Type*} 
  [ig : incidence_geometry Point Line] :
  ∀ l m : Line, l ≠ m → 
    (∃! A : Point, is_common_point A l m) 
    ∨ ¬ have_common_point Point l m := 
begin
  intros l m,
  contrapose,
  push_neg,
  rintro ⟨not_unique, hlm⟩,
  rw exists_unique at not_unique,
  push_neg at not_unique,
  cases hlm with A hA,
  rcases not_unique A hA with ⟨B, ⟨hB, hAB⟩⟩,
  rw ne_comm at hAB,
  exact unique_of_exists_unique (ig.I1 hAB) ⟨hA.1, hB.1⟩ ⟨hA.2, hB.2⟩
\end{lstlisting}
		\vspace{1em}
		\begin{lstlisting}
ig.I1 {A B : Point} (h : A ≠ B) : ∃! l : Line, A ~ l ∧ B ~ l
\end{lstlisting}
		\columnbreak
		\begin{lstlisting}
Point Line: Type u
ig: incidence_geometry Point Line
l m : Line
A B: Point
hA: is_common_point A l m
hB: is_common_point B l m
hAB: A ≠ B

⊢ l = m
\end{lstlisting}
	\end{multicols}
\end{frame}










\begin{frame}[fragile]
	\begin{multicols}{2}
		\begin{lstlisting}
lemma neq_lines_have_at_most_one_common_point 
  (Point : Type*) {Line : Type*} 
  [ig : incidence_geometry Point Line] :
  ∀ l m : Line, l ≠ m → 
    (∃! A : Point, is_common_point A l m) 
    ∨ ¬ have_common_point Point l m := 
begin
  intros l m,
  contrapose,
  push_neg,
  rintro ⟨not_unique, hlm⟩,
  rw exists_unique at not_unique,
  push_neg at not_unique,
  cases hlm with A hA,
  rcases not_unique A hA with ⟨B, ⟨hB, hAB⟩⟩,
  rw ne_comm at hAB,
  exact unique_of_exists_unique (ig.I1 hAB) ⟨hA.1, hB.1⟩ ⟨hA.2, hB.2⟩
\end{lstlisting}
		\vspace{1em}
		\begin{lstlisting}
ig.I1 hAB : ∃! l : Line, A ~ l ∧ B ~ l
\end{lstlisting}
		\columnbreak
		\begin{lstlisting}
Point Line: Type u
ig: incidence_geometry Point Line
l m : Line
A B: Point
hA: is_common_point A l m
hB: is_common_point B l m
hAB: A ≠ B

⊢ l = m
\end{lstlisting}
	\end{multicols}
\end{frame}










\begin{frame}[fragile]
	\begin{multicols}{2}
		\begin{lstlisting}
lemma neq_lines_have_at_most_one_common_point 
  (Point : Type*) {Line : Type*} 
  [ig : incidence_geometry Point Line] :
  ∀ l m : Line, l ≠ m → 
    (∃! A : Point, is_common_point A l m) 
    ∨ ¬ have_common_point Point l m := 
begin
  intros l m,
  contrapose,
  push_neg,
  rintro ⟨not_unique, hlm⟩,
  rw exists_unique at not_unique,
  push_neg at not_unique,
  cases hlm with A hA,
  rcases not_unique A hA with ⟨B, ⟨hB, hAB⟩⟩,
  rw ne_comm at hAB,
  exact unique_of_exists_unique (ig.I1 hAB) ⟨hA.1, hB.1⟩ ⟨hA.2, hB.2⟩
end
\end{lstlisting}
		\columnbreak
		\lstinline{goals accomplished} \checkmark
	\end{multicols}
\end{frame}























\end{document}

