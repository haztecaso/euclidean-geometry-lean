\subsection{Geometría de congruencia}

\subsubsection{Para segmentos}
\setaxsection{C}

\begin{ax}\label{C1}
  Dados un segmento $\overline{AB}$ y un rayo $r$ con vértice $C$, existe un
  único punto $D$ en el rayo $r$ tal que $\overline{AB}\cong\overline{CD}$.
\end{ax}

\begin{ax}\label{C2}
  Si $\overline{AB}\cong\overline{CD}$ y $\overline{AB}\cong\overline{EF}$
  entonces $\overline{CD}\cong\overline{EF}$. Además cada segmento es congruente
  con sí mismo.
\end{ax}

\begin{ax}[Suma]\label{C3}
  Dados tres puntos $A, B, C$ en una línea y tales que $A * B * C$ y otros tres
  puntos $D, E, F$ en una línea tales que $D * E * F$, si
  $\overline{AB}\cong\overline{DE}$ y $\overline{BC}\cong\overline{EF}$ entonces
  $\overline{AC}\cong\overline{DF}$.
\end{ax}

\subsubsection{Para ángulos}

\begin{ax}\label{C4}
  Dados un ángulo $\angle ABC$ y un rayo $\overrightarrow{DF}$, fijado un lado
  del plano de la línea $DF$, existe un único rayo $\overrightarrow{DE}$ en
  dicho lado tal que $\angle BAC\cong\angle EDF$.

\end{ax}

\begin{ax}\label{C5}
  Para cada tres ángulos $\alpha, \beta, \gamma$, si $\alpha\cong\beta$ y
  $\alpha\cong\gamma$ entonces $\beta\cong\gamma$. Además cada ángulo es
  congruente con sí mismo.
\end{ax}

\begin{ax}[SAS]\label{C6}
  Sean tres triángulos $ABC$ y $DEF$ tales que
  $\overline{AB}\cong\overline{DE}$, $\overline{AC}\cong\overline{DF}$ y $\angle
  BAC\cong\angle EDF$. Entonces los dos triángulos son congruentes, es decir
  $\overline{BC}\cong\overline{EF}$, $\angle ABC\cong\angle DEF$ y $\angle
  ACB\cong\angle DFE$.
\end{ax}

\lstleanfull{congruence_geometry/basic.lean}{25}{74}
