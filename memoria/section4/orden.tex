\subsection{Geometría del orden}
\setaxsection{B}

El segundo grupo de axiomas establece propiedades de la relación de
\textit{orden}, una relación ternaria entre puntos. Dados tres puntos $A, B, C$
escribiremos $A * B * C$ para indicar que están relacionados mediante la
relación de orden.
Los cuatro \textit{axiomas de orden} son los siguientes:

\begin{ax}\label{ax:B1}
	Si un punto $B$ está entre $A$ y $C$ ($A * B * C$) entonces $A, B, C$ son
	distintos, están en una misma línea y $C * B * A$.
\end{ax}

\begin{ax}\label{ax:B2}
	Para cada dos puntos distintos $A,B$ existe un punto $C$ tal que $A * B * C$.
\end{ax}

\begin{ax}\label{ax:3}
	Dados tres puntos distintos en una línea, uno y sólo uno de ellos está entre
	los otros dos.
\end{ax}

\begin{ax}[Pasch]\label{ax:4}
	Sean $A, B, C$ tres puntos no colineares y $l$ una línea que no contenga a
	ninguno de estos puntos. Si $l$ contiene un punto $D$ que está entre $A$ y $B$
	($A * D * B$) entonces también debe contener un punto entre $A$ y $C$ o un
	punto entre $B$ y $C$.
\end{ax}

\lstleanfull{order_geometry/basic.lean}{21}{33}

\todo{explicar extension de clases.}

\subsubsection*{Definiciones}

\begin{defin*}[Segmentos]
  Dados dos puntos distintos $A, B$ definimos el \textbf{segmento}
  $\overline{AB}$ como el conjunto de puntos que contiene a $A, B$ y a todos los
  puntos que están entre ellos. Diremos que $A$ y $B$ son los extremos del
  segmento $\overline{AB}$.
\end{defin*}

\lstleanfull{order_geometry/basic.lean}{52}{52}

\begin{defin*}[Pertenencia de puntos a segmentos]
	\todo{redactar}
\end{defin*}

\lstleanfull{order_geometry/basic.lean}{56}{59}
\lstleanfull{order_geometry/basic.lean}{63}{64}

\todo{Explicar cómo se puede usar la notación}

\begin{defin*}[Intersección entre segmentos y líneas]
	\todo{redactar}
\end{defin*}

\lstleanfull{order_geometry/basic.lean}{106}{108}

\begin{defin*}[Triángulos]
	\todo{redactar}
\end{defin*}

\lstleanfull{order_geometry/basic.lean}{111}{113}

\begin{defin*}[Rayos]
  Dados dos puntos distintos $A, B$, definimos el \textbf{rayo}
  $\overrightarrow{AB}$ como el conjunto que contiene a $A$ y a todos los puntos
  de la línea $AB$ tales que están del mismo lado de $A$ que $B$. Dado un rayo
  $\overrightarrow{AB}$ llamaremos \textbf{vértice} del rayo al punto $A$.
\end{defin*}

\lstleanfull{order_geometry/basic.lean}{143}{143}

\begin{defin*}[Ángulos]
  Un \textbf{ángulo} es la unión de dos rayos $\overrightarrow{AB}$ y
  $\overrightarrow{AC}$ con el mismo vértice y no contenidos en una misma recta.
  Denotaremos dicho ángulo por $\angle ABC$.
\end{defin*}

\lstleanfull{order_geometry/basic.lean}{155}{158}
\lstleanfull{order_geometry/basic.lean}{179}{187}

\begin{defin*}
  Diremos que dos puntos \textbf{están del mismo lado del plano} respecto de una
  recta si el segmento que los une no contiene ningún punto de la recta.

  Diremos que dos puntos están del mismo \textbf{lado de una recta} respecto de
  un punto si el segmento que los une no contiene a dicho punto.
\end{defin*}

\lstleanfull{order_geometry/basic.lean}{193}{194}

\begin{defin*}[Lados de una línea]
	\todo{redactar}
\end{defin*}

\lstleanfull{order_geometry/basic.lean}{221}{223}

\subsubsection*{Resultados}%

\todo{Escribir sobre el paper de Meikle y los descubrimientos sobre las
	demostraciones originales de Hilbert}

\lstleanfull{order_geometry/propositions.lean}{17}{20}


\begin{prop}
	La relación de estar del mismo lado del plano respecto de una recta es una
	relación de equivalencia.
\end{prop}

\todo{¿Incluir código? Anexo?}


% \lstleanfull{order_geometry/propositions.lean}{25}{31}
% \lstleanfull{order_geometry/propositions.lean}{34}{50}
% \lstleanfull{order_geometry/propositions.lean}{59}{197}









