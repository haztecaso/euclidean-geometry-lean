\subsection{Geometría de congruencia}

\setaxsection{C}

El tercer grupo de axiomas establece propiedades de las relaciones de
\textit{congruencia entre segmentos} y \textit{congruencia entre ángulos}. La
relación de \textit{congruencia entre segmentos} es una relación binaria entre
\textit{segmentos}. La relación de \textit{congruencia entre ángulos} es una relación binaria entre
\textit{ángulos}. Usaremos el mismo símbolo $\cong$ para denotar ambas
relaciones y por el contexto quedará claro a cual nos referimos.

De la misma forma que en la geometría del orden, queremos considerar también los
axiomas anteriores, por lo que extenderemos la clase \textit{order_geometry}.

\lstleanfull{congruence_geometry/basic.lean}{18}{21}


\subsubsection{Congruencia de segmentos}

\begin{ax}\label{C1}
	\todo{redactar}
	% C1 : Given a seg and two distinct points `A` `B`, we find uniquely find a point `A` on the
	% same side with `B` to `A` such that seg `A` `C` is congruent to the seg.
\end{ax}

\lstleanfull{congruence_geometry/basic.lean}{22}{26}

\begin{ax}\label{C2}
	Si $\overline{AB}\cong\overline{CD}$ y $\overline{AB}\cong\overline{EF}$
	entonces $\overline{CD}\cong\overline{EF}$. Además cada segmento es
	congruente con sí mismo.
\end{ax}

\lstleanfull{congruence_geometry/basic.lean}{27}{29}

\begin{ax}[Suma]\label{C3}
	Dados tres puntos $A, B, C$ en una línea y tales que $A * B * C$ y otros
	tres puntos $D, E, F$ en una línea tales que $D * E * F$, si
	$\overline{AB}\cong\overline{DE}$ y $\overline{BC}\cong\overline{EF}$
	entonces $\overline{AC}\cong\overline{DF}$.
\end{ax}

\lstleanfull{congruence_geometry/basic.lean}{30}{36}

\subsubsection{Congruencia de ángulos}

\begin{ax}\label{C4}
	\todo{redactar}
	Dado un ángulo $\alpha$ y punto distintos $A$ y $B$ existe un punto
	$C$ tal que $\angle CAB\cong \alpha$. Este punto $C$ está univocamente
	determinado si fijamos un lado de la línea $\overline{AB}$.
\end{ax}

\lstleanfull{congruence_geometry/basic.lean}{37}{41}

El punto \lstinline{Side} y la hipótesis \lstinline{hSide} se han utilizado para
fijar un lado de la línea $\overline{AB}$.

\begin{ax}\label{C5}
	Para cada tres ángulos $\alpha, \beta, \gamma$, si $\alpha\cong\beta$ y
	$\alpha\cong\gamma$ entonces $\beta\cong\gamma$. Además cada ángulo es
	congruente con sí mismo.
\end{ax}

\lstleanfull{congruence_geometry/basic.lean}{42}{44}

\begin{ax}[SAS]\label{C6}
	Sean tres triángulos $ABC$ y $DEF$ tales que
	$\overline{AB}\cong\overline{DE}$, $\overline{AC}\cong\overline{DF}$ y
	$\angle BAC\cong\angle EDF$. Entonces los dos triángulos son congruentes, es
	decir $\overline{BC}\cong\overline{EF}$, $\angle ABC\cong\angle DEF$ y
	$\angle ACB\cong\angle DFE$.
\end{ax}

\lstleanfull{congruence_geometry/basic.lean}{45}{57}

\subsubsection{Resultados}

En esta sección solo hemos formalizado las dos proposiciones que establecen que
las relaciones de \textit{congruencia de segmentos} y \textit{congruencia de
	ángulos} son relaciones de equivalencia. Los enunciados y demostraciones son
análogos, por lo que incluimos solamente el caso de los segmentos.


\lstleanfull{congruence_geometry/basic.lean}{61}{86}

