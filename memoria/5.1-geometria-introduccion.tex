\section{Formalizando la geometr\'{i}a de Hilbert en Lean}

En esta sección presentamos algunos axiomas y resultados elementales de la
axiomática de Hilbert, comparando los enunciados y demostraciones expresados de
forma natural con sus correspondientes formalizaciones en Lean.

En 1899 Hilbert empezó a desarrollar su propuesta de nueva fundamentación de la
geometría euclídea, mediante una serie de apuntes de conferencias que más tarde
se convertirían en el tratado \textit{Grundlagen der Geometrie} (Fundamentos de
Geometría). Dicho trabajo hace énfasis en los problemas de clasificación de
nociones primitivas, grupos de axiomas, interdependencias entre las distintas
partes de la teoría y minimalidad de los axiomas considerados. Esta nueva teoría
geométrica parte de postular ciertas nociones primitivas y una serie de axiomas
que establecen cómo se relacionan estas nociones.

En este trabajo hemos seguido una versión modernizada de los axiomas y los
resultados, basada en las presentaciones de los libros de
Hartshorne~\cite{hartshorneGeometryEuclid2000} y
Greenberg~\cite{greenbergEuclideanNonEuclideanGeometries1993}, en las que se
considera el caso restringido de la geometría plana.

Consideraremos por tanto las siguientes nociones primitivas: dos términos
primitivos (\textit{puntos} y \textit{líneas}) y cuatro relaciones primitivas
(\textit{incidencia}, \textit{orden}, \textit{congruencia de segmentos} y
\textit{congruencia de ángulos}).

Trataremos los axiomas y definiciones correspondientes a estas nociones
primitivas, siguiendo la estructura del tratado, mencionando alguna cuestión
sobre el axioma de las paralelas, pero sin entrar en cuestiones de continuidad.
Además se incluirá alguna formalización de resultados demostrables con las
nociones presentadas.

