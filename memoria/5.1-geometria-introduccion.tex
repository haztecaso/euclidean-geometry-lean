\section{Formalizando la geometría de Hilbert en Lean}
% \begin{itemize}
% 	\item Otros trabajos
% 	      \begin{itemize}
% 		      \item Descubrimiento de saltos de intuicion en Hilbert
% 		      \item Paper en el que se analizan las decisiones de diseño de
% 		            software a la hora de formalizar.
% 		      \item Formalización de la independencia del quinto postulado
% 		            utilizando los axiomas de Tarski.
% 	      \end{itemize}
% 	\item Mi trabajo. Formalizando la geometría de Hilbert.
% 	      \begin{itemize}

% 		      \item Geometría de incidencia. Comparación entre los
% 		            axiomas originales de Hilbert, su redacción moderna y
% 		            mi formalización en Lean. Introducción a conceptos
% 		            y funciones de Lean mediante ejemplos (clases,
% 		            tipos de parámetros, etc)
% 		      \item Otros grupos de axiomas y tratamientos.
% 		      \item Idea demostración de independencia del axioma de
% 		            las paralelas.
% 	      \end{itemize}
% \end{itemize}

En esta sección se presentan algunos axiomas y resultados elementales de la
axiomática de Hilbert, comparando los enunciados y demostraciones expresados de
forma natural con sus correspondientes formalizaciones en Lean.

En 1899 Hilbert empezó a desarrollar su propuesta de nueva fundamentación de la
geometría euclídea, mediante una serie de apuntes de conferencias que más tarde
se convertirían en el tratado \textit{Grundlagen der Geometrie} (Fundamentos de
Geometría). Este trabajo hace énfasis en los problemas de clasificación de
nociones primitivas, grupos de axiomas, interdependencias entre las distintas
partes de la teoría y minimalidad de los axiomas considerados.

Esta nueva teoría geométrica parte de postular ciertas nociones primitivas y una
serie de axiomas que establecen cómo se relacionan estas nociones. En este
trabajo se ha seguido una versión modernizada de los axiomas y los resultadas,
basada en las presentaciones de los libros de
\textit{Hartshorne}~\cite{hartshorneGeometryEuclid2000} y
\textit{Greenberg}~\cite{greenbergEuclideanNonEuclideanGeometries1993}, en las
que se considera el caso restringido de la geometría plana.

Consideraremos por tanto cinco nociones primitivas: dos términos primitivos
(\textit{puntos} y \textit{líneas}) y cuatro relaciones primitivas
(\textit{incidencia}, \textit{orden}, \textit{congruencia de segmentos} y
\textit{congruencia de ángulos}).

% \begin{itemize}
% 	\item Denotaremos los \textit{puntos} con letras mayúsculas $A,B,C,\dots$
% 	\item Denotaremos las \textit{líneas} con letras minúsculas $a,b,c,\dots$
% \end{itemize}

Se tratarán los axiomas y definiciones correspondientes a estas nociones
primitivas, siguiendo la estructura del tratado, mencionando alguna cuestión
sobre el axioma de las paralelas, pero sin entrar en cuestiones de continuidad.
Además se incluirá alguna formalización de resultados demostrables con las
nociones presentadas.


