\section{Objetivos y metodolog\'{i}a}

El objetivo principal de este trabajo es aprender los fundamentos del lenguaje
de programación y asistente de demostración Lean. Sin entrar en
detalles sobre cómo funciona y la implementación del lenguaje, nos hemos
propuesto aprender a utilizarlo como podría usarlo un estudiante o investigador de
matemáticas. Es decir, formalizando resultados matemáticos.

Para esto hemos elegido la geometría euclídea plana, en particular la
axiomatización de David Hilbert realizada a inicios del siglo pasado.
La idea es que esta teoría sirva de guión para formalizar resultados y durante
el proceso de formalización de axiomas y proposiciones aprender las
características y nociones necesarias de Lean.

Otro objetivo más ambicioso, que trasciende las posibilidades de este trabajo,
es el de formalizar la independencia del \textit{axioma de las paralelas} del
resto de axiomas de la geometría. Aun no teniendo el tiempo suficiente para
abordar esta tarea, hemos propuesto una formalización del enunciado de la
independencia y propuesto un plan de trabajo para formalizar su demostración.

Antes de poder emprender el trabajo de formalización de resultados geométricos
ha sido necesario obtener unos conocimientos de base de Lean. Al abordar esta
tarea, en la que aprendimos a expresar y demostrar resultados elementales de
lógica proposicional, a trabajar con estructuras algebraicas y exploramos
ciertas características del lenguaje, las principales referencias han sido los
materiales del curso de Kevin Buzzard, \textit{Formalising
	mathematics}~\cite{buzzardFormalisingMathematicsFormalising}, y el manual
oficial, \textit{Theorem proving in Lean}~\cite{avigadLeanTheoremProver}.

Los materiales del curso, que incluyen ejercicios prácticos, explicaciones
teóricas y soluciones a los ejercicios en formato vídeo han sido especialmente
útiles en este proceso de aprendizaje.

Durante el mes de febrero de 2023 tuvimos la oportunidad de participar en el
curso de doctorado sobre \textit{Formalización de matemáticas en Lean},
impartido por María Inés de Frutos~\cite{defrutosFormalizacionMatematicasLean}.
Asistir a este curso nos permitió afianzar los conocimientos que estábamos
adquiriendo y contrastar dudas o distintas formas de resolver los problemas con
la profesora y los demás participantes del curso.

Para tener un entendimiento a nivel intuitivo de los conceptos necesarios de
teoría de tipos se ha estudiado el primer capítulo del libro \textit{Homotopy
	Type Theory}~\cite{HomotopyTypeTheory} en el que se introducen conceptos de
teoría de tipos y se mencionan sus relaciones con la lógica.

En paralelo a este aprendizaje del lenguaje y herramientas de Lean iniciamos el
estudio de la geometría de Hilbert, apoyándonos principalmente en los libros de
texto de Hartshorne~\cite{hartshorneGeometryEuclid2000} y
Greenberg~\cite{greenbergEuclideanNonEuclideanGeometries1993}, y realizando
alguna consulta al texto original de Hilbert~\cite{hilbertFoundationsGeometry1950},
para conocer cómo fueron formulados originalmente los axiomas, definiciones y
teoremas de la teoría.

Una vez abordados los conceptos básicos necesarios de Lean y de la
geometría euclídea, emprendimos la tarea de la formalización. Para ello, el
proceso ha consistido fundamentalmente en seguir linealmente la teoría como está
presentada en los libros de referencia e intentar formalizar axiomas,
definiciones, proposiciones y demostraciones. A medida que hemos ido avanzando
y la dificultad de los resultados por formalizar ha ido creciendo, hemos tenido
que volver a las referencias de Lean, consultando documentación y
aprendiendo nuevas características del lenguaje.

El código completo que hemos desarrollado puede encontrarse en un repositorio
github. En el apéndice \ref{sec:repositorio} se encuentra el enlace y un esquema
de la estructura de los ficheros. En este documento, cada vez que incluimos
extractos de estos ficheros hemos referenciado las líneas incluidas y nombre del
fichero.

Durante este proceso de digitalicación hemos consultado trabajos similares que
analizan la geometría de Hilbert, uno realizado en el lenguaje
Isabelle/Isar~\cite{meikleFormalizingHilbertGrundlagen2003}, en el que
se analizan en detalle las demostraciones originales de Hilbert, y otro
que utiliza el lenguaje Coq/Gallina~\cite{dehlinger2001higher}, en el
que se exploran formalizaciones mediante una lógica intuicionista.






