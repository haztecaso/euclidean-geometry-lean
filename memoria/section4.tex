\section{Formalizando la geometría de Hilbert en Lean}
\begin{itemize}
	\item Otros trabajos
	      \begin{itemize}
		      \item Descubrimiento de saltos de intuicion en Hilbert
		      \item Paper en el que se analizan las decisiones de diseño de
		            software a la hora de formalizar.
		      \item Formalización de la independencia del quinto postulado
		            utilizando los axiomas de Tarski.
	      \end{itemize}
	\item Mi trabajo. Formalizando la geometría de Hilbert.
	      \begin{itemize}

		      \item Geometría de incidencia. Comparación entre los
		            axiomas originales de Hilbert, su redacción moderna y
		            mi formalización en Lean. Introducción a conceptos
		            y funciones de Lean mediante ejemplos (clases,
		            tipos de parámetros, etc)
		      \item Otros grupos de axiomas y tratamientos.
		      \item Idea demostración de independencia del axioma de
		            las paralelas.
	      \end{itemize}
\end{itemize}

En esta sección se presentan algunos axiomas y resultados elementales de la
axiomática de Hilbert, comparando los enunciados y demostraciones expresados de
forma natural (\todo{¿Cómo expresar esto bien?}) con sus correspondientes
formalizaciones en Lean.

\subsection{Geometría de incidencia}

\todo{Explicar cómo funcionan las clases, comparar los axiomas de hilbert
	expresados en lenguaje natural con sus formalizaciones. Diferencias entre
	parámetros implícitos y explícitos.}


\setaxsection{I}
\begin{ax}\label{I1}
	Para cada par de puntos distintos $A, B$ existe una única recta que los
	contiene.
\end{ax}

\begin{ax}\label{I2}
	Cada línea contiene al menos dos puntos distintos.
\end{ax}

\begin{ax}\label{I3}
	Existen tres puntos no colineares. Es decir, existen $A$, $B$ y $C$ tales que
	$AB\neq BC$.
\end{ax}

\begin{lstlisting}
class incidence_geometry (Point Line : Type*) :=
  (lies_on : Point → Line → Prop)
  (infix ` ~ ` : 50 := lies_on)
  (I1 {A B : Point} (h : A ≠ B): ∃! l : Line, A ~ l ∧ B ~ l)
  (I2 (l : Line) : ∃ A B : Point, A ≠ B ∧ A ~ l ∧ B ~ l)
  (I3 : ∃ A B C : Point, different3 A B C ∧ ¬ ∃ l : Line,  A ~ l ∧ B ~ l ∧ C ~ l)
\end{lstlisting}

\subsection{Resultados elementales}

Uno de los primeros resultados que se pueden demostrar, utilizando sólamente los
axiomas de incidencia es el siguiente:

\begin{prop}
	Dos líneas distintas pueden tener como mucho un punto en común.
\end{prop}


Su correspondiente formalización en Lean es:


\begin{lstlisting}
lemma disctinct_lines_one_common_point
  {Point Line : Type*} [ig : incidence_geometry Point Line] :
  ∀ l m : Line, l ≠ m →
    (∃! A : Point, is_common_point A l m) ∨ (¬ have_common_point Point l m) :=
begin
  sorry
end
\end{lstlisting}


Se puede observar que, gracias al uso de caracteres unicode, la formalización en
Lean es muy fácil de leer y cercana a la forma de escribir matemáticas a la que
estamos acostumbrados.

\todo{¿Incidir sobre la diferencia entre parámetros explícitos e implícitos?}

Observemos que en el enunciado se están utilizando algunas definiciones que se
han declarado previamente:

\begin{lstlisting}
def is_common_point {Point Line : Type*} [incidence_geometry Point Line]
  (A : Point) (l m : Line) := A ~ l ∧ A ~ m

def have_common_point (Point : Type*) {Line : Type*} [incidence_geometry Point Line]
  (l m : Line) := ∃ A : Point, is_common_point A l m
\end{lstlisting}

\todo{Aquí si que sería conveniente explicar por qué algunos parámetros son
	implícitos y otros explícitos.}

La demostración, como la presenta el libro de Hartshorne (\todo{citar}), es como
sigue:
\begin{proof}
	Sean $l$ y $m$ dos líneas. Supongamos que ambas contienen los puntos $A$ y
	$B$ con $A\ne B$. Por el axioma I1, existe una única línea que pasa por $A$
	y $B$, por lo que $l$ y $m$ deben ser iguales.
\end{proof}

Esta demostración puede interpretarse como una demostración por absurdo sobre la
condición de que las dos líneas sean iguales, o como una demostración por
contraposición: si asumimos que no se cumple la conclusión (que las dos lineas
no tengan más de un punto en común), entonces tampoco se cumple la premisa (que las
dos líneas sean iguales).

\todo{Explicar qué es una demostración en Lean. Que es el estado táctico y la
	meta, qué hacen las tácticas.}

Al intentar implementar esta idea en Lean nos damos cuenta de que hay bastantes
detalles que necesitamos tener en cuenta.

Esta es la formalización completa del resultado en Lean, incluyendo el enunciado
y su demostración:

\lstinputlisting[style=leanFull, caption={\lstinline{disctinct_lines_one_common_point}}]{listing1.lean}

Analicemos la demostración línea por línea:
\begin{enumerate}[label=L.\arabic*, topsep=0mm]
	\setcounter{enumi}{4}

	\item El estado táctico inicial incluye los parámetros del lema. En este caso
	      los tipos \lstinline{Point} y \lstinline{Line} e \lstinline{ig}, la
	      instancia de la clase \lstinline{incidence_geometry}. Esta instancia
	      representa el hecho de que los tipos \lstinline{Point} y \lstinline{Line}
	      cumplen los axiomas de la geometría de incidencia.

	      La meta se corresponde con el enunciado del lema, es decir lo que queremos
	      demostrar.


	\item \lstinline{intros l m,} La aplicación de la táctica \lstinline{intros}
	      introduce las hipótesis \lstinline{l} y \lstinline{m}. Es decir, saca el
	      cuantificador universal de la meta e introduce las variables cuantificadas
	      en el estado táctico, pasando a tener ahora dos nuevos términos
	      \lstinline{l : Line} y \lstinline{m : Line}. La nueva meta es
	      \begin{lstlisting}
l ≠ m → (∃! (A : Point), is_common_point A l m) ∨ ¬have_common_point Point l m
\end{lstlisting}
	      Esto equivale a decir en lenguaje natural "sean l y m dos líneas"

	\item \lstinline{contrapose,} La táctica \lstinline{contrapose} permite
	      realizar una demostración por contraposición. Es decir, si nuestra meta es
	      de la forma \lstinline{A → B}, la reemplaza por \lstinline{¬B → ¬A}. En
	      este caso la meta resultante es
	      \begin{lstlisting}
⊢ ¬((∃! (A : Point), is_common_point A l m) ∨ ¬have_common_point Point l m) → ¬l ≠ m
\end{lstlisting}

	\item \lstinline{push_neg, } La táctica \lstinline{push_neg} utiliza
	      equivalencias lógicas para \guillemotleft empujar\guillemotright las negaciones dentro de la fórmula.
	      En este caso, al no haber especificado una hipótesis concreta, se aplica
	      sobre la meta.

	      En la primera parte de la implicación se aplica una ley de De Morgan para
	      introducir la negación dentro de una disjunción, convirtiéndola en una
	      conjunción de negaciones. En la segunda, negar una desigualdad equivale a
	      una igualdad.

	      \todo{fix latex problem}
	      % Por tanto la meta resultante es
	      % \begin{lstlisting}
	      % ⊢ (¬∃! (A : Point), is_common_point A l m) ∧ have_common_point Point l m → l = m
	      % \end{lstlisting}

	      Es interesante notar que \lstinline{push_neg} no consigue 'empujar' la negación todo lo que podríamos desear.

	      Esto es así porque no está reescribiendo las definiciones previas y de
	      \lstinline{∃!}. Esto lo tendremos que hacer manualmente, como se verá
	      enseguida.

	\item \lstinline{rintro ⟨not_unique, hlm⟩,} La táctica \lstinline{rintro}
	      funciona como \lstinline{intro}, en este caso aplicada para asumir la
	      hipótesis de la implicación que queremos demostrar. La variante
	      \lstinline{rintro} nos permite entrar en definiciones recursivas, en este
	      caso en la del operador \lstinline{∧}, y mediante el uso de los paréntesis
	      \lstinline{⟨⟩} introducir los dos lados de la conjunción como hipótesis
	      separadas. Por tanto después de aplicar esta táctica obtendremos dos
	      hipótesis adicionales:
	      \begin{lstlisting}
not_unique: ¬∃! (A : Point), is_common_point A l m
hlm: have_common_point Point l m
\end{lstlisting}
	      y la meta resultante es el segundo lado de la implicación, es decir
	      \lstinline{⊢ l = m}.

	\item \lstinline{rw exists_unique at not_unique,}  La táctica \lstinline{rw}
	      (abreviación de \lstinline{rewrite}) nos permite reescribir ocurrencias de
	      fórmulas utilizando definiciones o lemas de la forma \lstinline{A ↔ B}. Al
	      escribir \lstinline{at} indicamos dónde queremos realizar dicha
	      reescritura, en este caso en la hipótesis \lstinline{not_unique}.

	      En este caso utilizamos la definición de \lstinline{∃!}, con lo que se modifica la hipótesis
	      \begin{lstlisting}
not_unique : ¬∃ (x : Point), 
is_common_point x l m ∧ ∀ (y : Point), is_common_point y l m → y = x
\end{lstlisting}

	\item \lstinline{push_neg at not_unique,}
	      %   /- Como comentábamos antes, ahora que hemos reescrito la definición de ∃! podemos seguir "empujando" la negación en la hipótesis not_unique,
	      %      obteniendo ∀ (x : Point), is_common_point x l m → (∃ (y : Point), is_common_point y l m ∧ y ≠ x)
	      %   -/

	\item \lstinline{cases hlm with A hA,} La táctica \lstinline{cases} nos
	      permite, entre otras cosas, dada una hipótesis de existencia, obtener un
	      término del tipo cuantificado por el existe y la correspondiente hipótesis
	      particularizada para el nuevo término.

	      En nuestro caso tenemos la hipótesis \lstinline{hlm: have_common_point Point l m} y la definición
	      \lstinline{have_common_point Point l m := ∃ A : Point, is_common_point A l m}.

	      Por tanto al aplicar la táctica, la hipótesis hlm se convierte en dos
	      nuevas hipótesis
	      \begin{lstlisting}
A : Point 
hA: is_common_point A l m
\end{lstlisting}

	\item \lstinline{rcases not_unique A hA with ⟨B, ⟨hB, hAB⟩⟩,} En esta línea
	      están ocurriendo distintas cosas:
	      \begin{itemize}
		      \item Recordemos que en el estado táctico actual tenemos la hipótesis
		            \begin{lstlisting}
not_unique: ∀ (x : Point), is_common_point x l m 
→ (∃ (y : Point), is_common_point y l m ∧ y ≠ x) 
\end{lstlisting}

		            Primero se está construyendo el término \lstinline{not_unique A hA},
		            al que posteriormente se le aplicará la táctica {rcases}.

		            En Lean los cuantificadores universales y las implicaciones pueden
		            tratarse como funciones. Al pasar el primer argumento \lstinline{A}
		            estamos particularizando la cuantificación sobre el punto
		            \lstinline{x}, proporcionando el término \lstinline{A : Point} que
		            tenemos entre nuestras hipótesis. Por tanto el término
		            \lstinline{not_unique A} es igual a
		            \begin{lstlisting}
is_common_point A l m → (∃ (y : Point), is_common_point y l m ∧ y ≠ A)
\end{lstlisting}

		            Ahora podemos observar que tenemos entre nuestras hipótesis la
		            condición de esta implicación, \lstinline{hA: is_common_point A l m}
		            Al pasar este término como segundo argumento obtenemos la conclusión
		            de la implicación, y por tanto el término
		            \lstinline{not_unique A hA} es igual a
		            \begin{lstlisting}
∃ (y : Point), is_common_point y l m ∧ y ≠ x
\end{lstlisting}

		      \item La aplicación de la táctica \lstinline{rcases} nos permite, como
		            anteriormente, obtener un término concreto del cuantificador
		            existencial y además profundizar en la definición recursiva del
		            \lstinline{∧}, generando así dos hipótesis separadas. Obtenemos por
		            tanto las nuevas hipótesis
		            \begin{lstlisting}
B: Point 
hB: is_common_point B l m 
hAB: B ≠ A
\end{lstlisting}
	      \end{itemize}

	\item \lstinline{rw ne_comm at hAB,} Para tener la hipótesis
	      \lstinline{hAB: B ≠ A} en el mismo orden que el utilizado en los axiomas y
	      poder utilizarlos correctamente, reescribimos la hipótesis \lstinline{hAB}
	      utilizando la propiedad conmutativa de la desigualdad, obteniendo así la
	      hipóteis \lstinline{hAB: A ≠ B}.

	\item \lstinline{exact unique_of_exists_unique (ig.I1 hAB) ⟨hA.left,hB.left⟩ ⟨hA.right,hB.right⟩,}

	      La táctica \lstinline{exact} se utiliza para concluir la demostración
	      proporcionando un término igual a la meta. Recordemos que la meta actual
	      es \lstinline{⊢ l = m}.

	      Analicemos entonces el término que estamos proporcionando a la táctica.

	      El lema \lstinline{unique_of_exists_unique}, definido en la librería
	      estándar de Lean, sirve para extrer la parte de unicidad del cuantificador
	      \lstinline{∃!}. Dadas una fórmula de la forma \lstinline{∃! x, px} y dos
	      fórmulas \lstinline{p a} y \lstinline{p b} , devuelve la fórmula que
	      aserta la igualdad entre los términos que cumplen la propiedad
	      \lstinline{p}: \lstinline{a = b}.

	      Como primer argumento le estamos pasando el primer axioma de incidencia,
	      particularizado con la hipóteis \lstinline{hAB : A ≠ B}. Es decir
	      \lstinline{ig.I1 hAB} es igual a \lstinline{∃! l : Line, A ~ l ∧ B ~ l}.

	      Ahora queremos pasar en los otros dos argumentos términos
	      \lstinline{A ~ l ∧ B ~ l} y \lstinline{A ~ m ∧ B ~ m}, para obtener la
	      igualdad \lstinline{l = m}. Para esto tenemos que recombinar las hipótesis
	      \lstinline{hA} y \lstinline{hB}.

	      \lstinline{hA.left} es igual a \lstinline{A ~ l} y \lstinline{hB.left} a
	      \lstinline{B ~ l}, y mediante los paréntesis \lstinline{⟨⟩} combinamos
	      estos términos en la conjunción \lstinline{⟨hA.left,hB.left⟩}, obteniendo
	      \lstinline{A ~ l ∧ B ~ l}.

	      El uso de los paréntesis nos permite construir una conjunción sin tener
	      que especificar esplícitamente que queremos construir una conjunción, pero
	      el sistema de tipos de Lean permite inferir que el término esperado es una
	      conjunción.

	      Análogamente para el segundo argumento.
\end{enumerate}





















