\section{Conclusiones}

La lectura del artículo de \textit{Meikle} y
\textit{Fleuriot}~\cite{meikleFormalizingHilbertGrundlagen2003} nos ha resultado
sumamente interesante. Lo consideramos una prueba más de que sistemas de
verificación formal como \textit{Lean} tienen mucho que aportar a las matemáticas
y nos ayudan a conocer como razonaban grandes matemáticos.

Encontrando detalles no totalmente justificados en demostraciones escritas hace
más de un siglo por \textit{Hilbert} descubrimos que el objetico de
\textit{Hilbert} de separar el trabajo deductivo formal de la intuición
geométrica no es realizable por completo. Hilbert utilizó razonamientos
en los que aparantemente cada paso estaba fundado exclusivamente en el uso de
una lógica precisa y los pasos anteriores. Pero en realidad ciertos pasos
resultan que pueden resultar evidentes si pensamos en el dibujo correspondiente,
necesitan ser desarrollados. No se pueden separar totalmente la intuición
matemática y su formalizacion.

Las herramientas de formalización en cierto sentido cumplen el proyecto de
formalización empezado por \textit{Hilbert}, mostrando que se puede llevar la
formalización hasta su extremo, comprobando automáticamente cada detalle. Pero a
su vez muestran sus propias limitaciones. De nada sirve la formalización si no
tenemos un contenido matemático que formalizar y una intuición con la que
interpretar y aprender de la nueva información que aportan sistemas como
\textit{Lean}.

Hemos aprendido también que no todo el trabajo está en desarrollar código en
\textit{Lean}. A veces no es inmediato cómo formalizar una definición o
proposición en Lean. La forma más intuitiva y práctica en matemáticas no tiene
por qué tener una correspondencia directa con la formalización que tenga más
sentido realizar en \textit{Lean}. Es necesario conocer muy bien las teorías,
conceptos y definiciones, que se están formalizando, a un nivel de abstracción
suficientemente amplio como para ser capaces de desarrollar formulaciones
equivalentes más convenientes para sus formalizaciones.

Más de una vez, a la hora de formalizar una demostración de una proposición de
la geometría euclídea en \textit{Lean} hemos descubierto que no habíamos
entendido por completo la demostración. \textit{Lean} ayuda enormemente a aislar
e identificar las partes de una demostración que no comprendemos. Si no
entendemos por completo un paso o una parte de una demostración no sabremos
formalizarla, e incluimos un \lstinline{sorry} en el código. A partir de aquí
entramos en un proceso iterativo en el que volvemos al texto de la prueba para
enfocarnos en los detalles restantes, y poco a poco vamos elimiando los
\lstinline{sorry}s.

En las formalizaciones realizadas en este trabajo no se hemos pretendido
alcanzar una síntesis y claridad óptima en las demostraciones, sino conseguir
que el sistema las acepte. Muchas de las pruebas pueden ser acortadas,
por ejemplo extrayendo partes a lemas adicionales para ser reutilizadas.
Seguramente no hayamos utilizado siempre las tácticas más adecuadas, habiendo
otras posibilidades más elegantes y claras de leer.

Hemos digitalizado solo una pequeña parte de la geometría de Hilbert. Invitamos
a los lectores interesados en este trabajo a participar en él, añadiendo los
axiomas y definiciones que faltan, demostrando resultados o proponiendo cambios.
Para ello se pueden crear \textit{tíquets} o \textit{merge requests} en el
repositorio de \textit{github}, enlazado en el anexo~\ref{sec:repositorio}.

Las principales dificultades han surgido a la hora de intentar formalizar
ejemplos concretos, como modelos de los distintos grupos de axiomas que se
consideran. Por ejemplo no he conseguido terminar de formalizar el modelo más
simple que he encontrado de una geometría de incidencia en el que no se cumple
el axioma de las paralelas.

En parte atribuimos estas dificultades a la falta de una documentación más
completa y accesible de la librería \textit{mathlib}, en la que se explique cómo
están implementados ciertos objetos matemáticos en ella y qué utilidades se
proporcionan para trabajar con ellos.

Al no haber estudiado en detalle el funcionamiento del lenguaje \textit{Lean} y
la estructura de la librería mathlib, nos hemos encontrado más de una vez con
errores o comportamientos que no hemos sabido interpretar. De nuevo los
problemas más recurrentes han surgido de las dificultades encontradas al
intentar entender el código disponible en la librería \textit{mathlib}.



