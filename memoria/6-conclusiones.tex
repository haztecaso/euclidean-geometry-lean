\section{Conclusiones}

- No todo el trabajo está en Lean. A veces no es inmediato cómo formalizar una
proposición en Lean. Hay que tener claros los conceptos y conocer muy bien qué es lo que se
está formalizando.

- Enunciados aparente muy simples pueden ser difíciles y largos de demostrar.

- No se ha pretendido alcanzar una síntesis y claridad en las demostraciones,
sino conseguir que el sistema las acepte. Probablemente muchos de los resultados
pueden ser

- Lean es muy avanzado y en diversas ocasiones nos hemos encontrado con errores
o comportamientos que no hemos sabido interpretar, debido a

- Las mayores dificultades han surgido a la hora de intentar formalizar ejemplos
concretos, como modelos de los distintos grupos de axiomas que se consideran.
Por ejemplo no he conseguido terminar de formalizar el modelo más simple que he
encontrado de una geometría de incidencia en el que no se cumple el axioma de
las paralelas.

En parte atribuyo estas dificultades a la falta de una documentación más
accesible de la librería \textit{mathlib} y cómo están implementados ciertos
objetos matemáticos en ella.

Para usar Lean a un nivel un poco más avanzado que el de este trabajo es
inevitable entrar en ciertos detalles de la teoría de tipos \textit{Calculus of
	Inductive Constructions} y cómo está implementada en \textit{Lean}.


\todo{Redactar}


