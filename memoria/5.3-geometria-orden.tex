\subsection{Geometría del orden}
\setaxsection{B}

El segundo grupo de axiomas establece propiedades de la relación de
\textit{orden}, una relación ternaria entre puntos. Dados tres puntos $A, B, C$
escribiremos $A * B * C$ para indicar que están relacionados mediante la
relación de orden.
Los cuatro \textit{axiomas de orden} son los siguientes:

\begin{ax}\label{ax:B1}
	Si un punto $B$ está entre $A$ y $C$ ($A * B * C$) entonces $A, B, C$ son
	distintos, están en una misma línea y $C * B * A$.
\end{ax}

\begin{ax}\label{ax:B2}
	Para cada dos puntos distintos $A,B$ existe un punto $C$ tal que $A * B * C$.
\end{ax}

\begin{ax}\label{ax:3}
	Dados tres puntos distintos en una línea, uno y sólo uno de ellos está entre
	los otros dos.
\end{ax}

\begin{ax}[Pasch]\label{ax:4}
	Sean $A, B, C$ tres puntos no colineares y $l$ una línea que no contenga a
	ninguno de estos puntos. Si $l$ contiene un punto $D$ que está entre $A$ y $B$
	($A * D * B$) entonces también debe contener un punto entre $A$ y $C$ o un
	punto entre $B$ y $C$.
\end{ax}

\lstleanfull{order_geometry/basic.lean}{21}{33}

\todo{explicar extension de clases.}

\subsubsection{Definiciones}

La definición usual de segmentos es la siguiente:

\begin{defin*}[Segmentos]
	Dados dos puntos distintos $A, B$ definimos el \textbf{segmento}
	$\overline{AB}$ como el conjunto de puntos que contiene a $A, B$ y a todos los
	puntos que están entre ellos.
\end{defin*}

Pero en nuestro caso no queremos utilizar teoría de conjuntos y por tanto
queremos evitar nociones como \guillemotleft conjunto de puntos \guillemotright.
Por tanto decimos simplemente que

\begin{defin*}[Segmentos]
	Dos puntos distintos $A, B$ determinan el \textbf{segmento} $\overline{AB}$.
	Diremos que $A$ y $B$ son los \textbf{extremos} del segmento $\overline{AB}$.
\end{defin*}

Esta definición es suficiente para determinar un segmento, pero para establecer
la relación de pertenencia de puntos a segmentos tendremos que complementarla
con otra definición. Para formalizar esta noción utilizamos una estructura:

\lstleanfull{order_geometry/basic.lean}{52}{52}

Cuando se define una estructura en \textit{Lean} se genera de forma automática
un constructor, una función útil para crear términos con tipo la estructura
definida. Dicha función se llamará como la estrucura más la notación
\lstinline{.mk} adjuntada: en este caso se tiene la función
\lstinline{Seg.mk : (neq : A ≠ B) → Seg Point} (observemos cómo los términos
\lstinline{A B : Point} son implícitos en la definición de la estructura).

\begin{defin*}[Pertenencia de puntos a segmentos]
	Decimos que un punto $A$ pertenece a un segmento $\overline{BC}$ si $A$ es
	igual a uno de los extremos del segmento ($B$ o $C$) o está entre ellos
	($B * A * C$).
\end{defin*}

\lstleanfull{order_geometry/basic.lean}{56}{59}

Esta función \lstinline{Seg.in} está dentro del espacio de nombres definido en
la estructura \lstinline{Seg}. Gracias a esto, si tenemos un término de la
estructura \lstinline{seg : Seg Point}, podremos llamar a la función utilizando
la notación de punto \lstinline{seg.in Line P}, en la que el segmento
\lstinline{seg} es pasado como primer argumento a la función \lstinline{seg}.
Por tanto con esta definición y la notación del punto estamos escribiendo la
pertenencia de puntos a segmentos en el orden contrario al usual:
\lstinline{seg.in Line P} significa que el punto \lstinline{P} pertenece al
segmento \lstinline{seg}. Esta notación del punto también permite acceder a los
elementos de una estructura (si consideramos la estructura como un producto
cartesiano de los tipos que la forman, este acceso se puede ver como una
proyección): En la definición de \lstinline{Seg.in} se puede ver como accedemos
a los extremos mediante la notación \lstinline{seg.A}, \lstinline{seg.B}.

Como en el caso de los segmentos, las siguientes definiciones hacen referencia
a nociones propios de la teoría de conjuntos, como uniones de segmentos.

En nuestra formalización hemos evitado dichas nociones, por lo que daremos las
definiciones adaptadas, que utilizan sólamente los tipos definidos
anteriormente.

\begin{defin*}[Triángulos]
	Tres puntos no colineares (por tanto distintos) $A$, $B$ y $C$, determinan
	el \textbf{triángulo} $\triangle ABC$. Los puntos $A$, $B$ y $C$ se llaman
	\textbf{vértices} del triángulo.
\end{defin*}

\lstleanfull{order_geometry/basic.lean}{111}{113}

En este caso también podemos definir funciones mediante la notación de punto que
nos permitan recuperar propiedades sobre una estructura, como por ejemplo la
propiedad de que los vértices del triángulo son distintos:

\lstleanfull{order_geometry/basic.lean}{136}{139}

\begin{defin*}[Rayos y pertenencia de puntos a rayos]

	Decimos que dos puntos distintos $A, B$ definen el \textbf{rayo}
	$\overrightarrow{AB}$. Dado un rayo $\overrightarrow{AB}$ llamaremos
	\textbf{vértice} del rayo al punto $A$.

	Decimos que un punto $P$ \textbf{pertenece} a un rayo $\overrightarrow{AB}$
	si

\end{defin*}

\lstleanfull{order_geometry/basic.lean}{143}{143}
\lstleanfull{order_geometry/basic.lean}{146}{151}

\begin{defin*}[Ángulos]
	Un \textbf{ángulo} es la unión de dos rayos $\overrightarrow{AB}$ y
	$\overrightarrow{AC}$ con el mismo vértice y no contenidos en una misma recta.
	Denotaremos dicho ángulo por $\angle ABC$.
\end{defin*}

\lstleanfull{order_geometry/basic.lean}{155}{158}
\lstleanfull{order_geometry/basic.lean}{179}{187}

\begin{defin*}
	Decimos que dos puntos \textbf{están del mismo lado del plano} respecto de una
	recta si el segmento que los une no contiene ningún punto de la recta.
\end{defin*}

\lstleanfull{order_geometry/basic.lean}{193}{194}

\begin{defin*}[Lados de una línea]
	Decimos que dos puntos están del mismo \textbf{lado de una línea} respecto de
	un punto si el segmento que los une no contiene a dicho punto.
\end{defin*}

\lstleanfull{order_geometry/basic.lean}{221}{223}


\subsubsection{Resultados}%

\todo{Escribir sobre el paper de Meikle y los descubrimientos sobre las
	demostraciones originales de Hilbert}

\lstleanfull{order_geometry/propositions.lean}{17}{20}


\begin{prop}
	La relación de estar del mismo lado del plano respecto de una recta es una
	relación de equivalencia.
\end{prop}

\todo{¿Incluir código? Anexo?}


% \lstleanfull{order_geometry/propositions.lean}{25}{31}
% \lstleanfull{order_geometry/propositions.lean}{34}{50}
% \lstleanfull{order_geometry/propositions.lean}{59}{197}




















