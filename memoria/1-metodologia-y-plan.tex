\section{Objetivos y metodología}

El objetivo principal de este trabajo es aprender los fundamentos del lenguaje
de programación y asistente de demostración \textit{Lean}. Sin entrar en
detalles sobre cómo funciona y está implementado el lenguaje, nos hemos
propuesto aprender a utilizarlo como podría usarlo un estudiante o investigador de
matemáticas. Es decir, formalizando resultados matemáticos.

Para esto hemos elegido la geometría euclídea plana, en particular la
axiomatización de \textit{David Hilbert} realizada a inicios del siglo pasado.
La idea es que esta teoría sirva de guión para formalizar resultados y durante
el proceso de formalización de axiomas y proposiciones aprender las
características y nociones necesarias de Lean para poder formalizarlos.

Otro objetivo más ambicioso, que trasciende las posibilidades de este trabajo,
es el de formalizar la independencia del \textit{axioma de las paralelas} del
resto de axiomas de la geometría. Aún no teniendo el tiempo suficiente para
abordar esta tarea, hemos propuesto una formalización del enunciado de la
independencia y propuesto un plan de trabajo para formalizar su demostración.

% Antes de poder emprender la formalización de la geometría estudiamos los

\todo{
	\begin{itemize}[topsep=0pt]
		\item Cursos y aprendiendo Lean. Formalising maths y curso María Inés
		\item Aprendizaje básico sobre teoría de tipos. HOTT
		\item Aprendiendo geometria, libros
		\item Lectura de trabajos similares
		\item Implementación de los resultados, evitando el uso de la teoría de conjuntos.
		\item Referenciar apéndice \ref{sec:repositorio} en el que se explica cómo está
		      organizado el repositorio de código
	\end{itemize}
}


