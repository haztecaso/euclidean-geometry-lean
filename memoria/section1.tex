\section{Matemáticas y geometría formales.}

La formalización matemática es el proceso de representar enunciados y
demostraciones matemáticas utilizando un lenguaje formal y un conjunto de reglas
lógicas bien definidas. En este contexto, se establece un conjunto de
proposiciones fundamentales, conocidas como axiomas, que se aceptan sin requerir
justificación. Estos axiomas se manipulan y transforman mediante la aplicación
sistemática de las reglas lógicas para derivar nuevas proposiciones matemáticas.
El proceso de aplicar secuencialmente estas reglas lógicas para obtener una
conclusión a partir de proposiciones más básicas es conocido como demostración.


En la historia de las matemáticas los \textit{Elementos} de \textit{Euclides},
tratado matemático compuesto de trece libros escrito en el siglo III a.C., son
el ejemplo más antiguo de proyecto de formalización matemática. En la obra se
trata rigurosamente una extensa variedad de temas, como la geometría plana y
espacial o la teoría de números. Euclides es el primer autor en presentar los
conocimientos matemáticos siguiendo un método formal. En los tratados se
presentan los argumentos a partir de una serie de postulados, definiciones y
nociones comunes a partir de los cuales se demuestran proposiciones y teoremas
mediante razonamientos deductivos.

La obra de Euclides ha tenido una profunda influencia en las matemáticas, la
lógica y la filosofía. Los \textit{Elementos} se mantuvieron como la principal
referencia en geometría durante casi dos milenios. Las pruebas rigurosas,
apoyadas en el razonamiento lógico y la estructura del tratado establecieron un
estándar para la argumentación y la exposición matemática que todavía se
conserva en la actualidad.

A lo largo de la historia, se ha evidenciado que los \textit{Elementos} de
\textit{Euclides} no se ciñen estrictamente al método axiomático. En el tratado
se encuentran razonamientos sustentados en intuiciones geométricas y
construcciones con regla y compás, en lugar de en deducciones estrictamente
lógicas derivadas de los postulados.

Durante siglos, matemáticos y filósofos han examinado y escrutado la obra de
Euclides, identificando errores y omisiones y planteando cuestiones sobre la
relación entre los postulados. El quinto postulado de Euclides, también conocido
como postulado de las paralelas, ha sido sometido a un análisis riguroso. Este
postulado afirma que, dada una línea recta y un punto fuera de ella, existe
únicamente una línea recta paralela a la línea dada que pasa por el punto en
cuestión. Por siglos, numerosos matemáticos, intuyendo la innecesariedad del
postulado, intentaron demostrarlo a partir de los otros cuatro, pero sin éxito.
El problema fue resuelto en el siglo XIX con la concepción de nuevas geometrías,
como la hiperbólica y la elíptica, en las cuales el postulado de las paralelas
no se verifica. Quedó así demostrada la independencia del quinto postulado
respecto a los primeros cuatro, y por tanto su necesidad para la construcción de
la geometría euclídea.

Durante el siglo XIX se dieron avances trascendentales en el desarrollo de las
matemáticas y la lógica formales. Ejemplos notables son la
formulación del álgebra de Boole, la lógica de predicados propuesta por
\textit{Gottlob Frege} o el desarrollo de la \textit{aritmética de Peano}.
En este contexto el matemático alemán \textit{David Hilbert} publicó su obra
\textit{Grundlagen der Geometrie} (Fundamentos de Geometría), en la cual se
lleva a cabo una revisión exhaustiva de los \textit{Elementos}, planteando una
nueva axiomatización para formalizar correctamente los resultados de la
geometría euclídea, eliminando por completo el recurso a la intuición y
razonamientos geométricos en la presentación de los resultados.


\redactar{Explicar que este es el punto de partida en el que enmarcar este
	trabajo. A partir de aqui se va a explicar  qué aporta la formalización
	matemática asistida por computadores y en que sentido es un paso más.}




