\section{Formalización asistida por computadores}

En la docencia e investigación en matemáticas es usual escribir las
demostraciones utilizando el lenguaje natural. Se asume que cada paso podría
escribirse mediante fórmulas lógicas, en cuyo caso podría comprobarse que cada
paso deriva de los anteriores y la aplicación de las reglas lógicas. Este
trabajo es demasiado tedioso para los matemáticos, y el exceso
de detalles puede restar protagonismo a las ideas que se quieren exponer.

Sin embargo esta es una tarea perfecta para ser realizada de forma automática
por máquinas. Los asistentes de demostración nos permiten formalizar
definiciones, enunciados de teoremas, sus demostraciones, y verficar la
corrección de estas demostraciones. Algunos de estos sistemas son \textit{Coq},
\textit{Isabelle/HOL},  \textit{Agda}, \textit{Metamath}, \textit{Mizar} y
\textit{Lean}.

Además de un lenguaje, estos sistemas suelen constar de un entorno de desarrollo
que aporta información contextual útil en el proceso de demostración.
\textit{Lean} es especialmente popular debido a la calidad y facilidad de uso de
este entorno y a la existencia de \textit{mathlib}, una amplia librería de
matemáticas formalizadas.

Formalizar matemáticas consiste en digitalizar enunciados y resultados
escribiéndolos en un lenguaje de programación que garantiza, mediante una
correspondencia entre una teoría de tipos y la lógica con el lenguaje de la
teoría de conjuntos, la validez de cada paso. 

En lo que sigue de esta sección expondremos algunos beneficios y aplicaciones
de los asistentes de demostración. La conferencia \textit{¿Por qué formalizar
	matemáticas?} impartida por \textit{María Inés de Frutos} en la facultad de
Ciencias Matemáticas de la UCM ha sido la principal referencia a la hora de
elaborar esta lista~\cite{defrutosPorQueFormalizar2023a}.

\paragraph{Comprobación mecanizada de demostraciones.}

Estos sistemas pueden verificar la corrección de demostraciones largas y
técnicas, reduciéndolas a las aplicaciones fundamentales de las reglas del
sistema formal utilizado, como una teoría de tipos.

Un ejemplo en el que se ha conseguido formalizar un resultado muy avanzado es el
del \textit{Liquid Tensor Experiment}. En el año 2020 el medallista Fields
\textit{Peter Scholze} lanzó a la comunidad de formalización matemática el reto
de verificar un importante teorema publicado por él y \textit{Dustin
	Clausen}~\cite{scholzeLiquidTensorExperiment2022}. Peter explica que, antes de
que se formalizara el resultado, no estaba totalmente seguro de la corrección de
una parte técnica de la demostración, que contenía muchos detalles técnicos que
la comunidad no estaba estudiando.

La guía de esta demostración, que se terminó de formalizar en el año 2022, se
encuentra en formato web~\cite{scholzeBlueprintLiquidTensor} y es un buen
ejemplo de un documento que presenta resultados paralelamente en un formato
textual, legible por matemáticos, y su formalización en el lenguaje
\textit{Lean}.

Una de las visiones a futuro de los impulsores de \textit{Lean}, como
\textit{Kevin Buzzard}, es la idea de a cada artículo matemático se le asocie su
formalización y por tanto el papel de los revisores pase a enfocarse en el
interés de los resultados, puesto que su corrección estaría garantizada por el
asistente de demostración. Actualmente este objetivo está muy lejos de ser
alcanzado.

\paragraph{Digitalización de definiciones y enunciados.}

La creación de una base de datos digital puede mejorar la capacidad de búsqueda
de resultados matemáticos previos, convirtiéndose en un gran apoyo en el proceso
de investigación. El proyecto \textit{Formal Abstracts}~\cite{halesFormalAbstracts}
se propone realizar una enciclopedia matemática digital utilizando el lenguaje
\textit{Lean}, con el objetivo de formalizar la mayor cantidad posible de
teoremas publicados en un formato entendible a la vez por humanos y máquinas.
Además se quieren enlazar los términos utilizados en dichos enunciados con sus
definiciones precisas.

El proyecto todavía se encuentra en fase de diseño y queda bastante trabajo para
que alcance sus objetivos, pero muestra una dirección de trabajo muy
interesante.

\paragraph{Uso en docencia.}

En ciertos centros universitarios ya se imparten cursos apoyándose en el uso del
asistente \textit{Lean}. El principal impulsor del uso de estas herramientas en
la docencia es \textit{Kevin Buzzard}, que desde el año 2021 imparte el curso
\textit{Formalising mathematics} en el \textit{Imperial
	College}~\cite{buzzardFormalisingMathematicsFormalising}.

El uso de asistentes de demostración en la docencia fija una referencia objetiva
sobre qué es una demostración correcta de un resultado, obligando a los
estudiantes a ser precisos en sus razonamientos. Además los estudiantes pueden
aprovechar el entorno interactivo para comprobar autónomamente si sus
razonamientos son correctos, sin necesidad de esperar la corrección del
profesor.

El principal problema actual en este ámbito es que la barrera de entrada al uso
de estos asistentes es alta, puesto que la instalación de las herramientas y
librerías no es simple y es necesaria cierta comodidad para escribir código.



\paragraph{Demostración automatizada.} Se espera que en un
futuro se puedan incorporar a estos sistemas téncicas de inteligencia artificial
que proporcionen indicaciones y ayuda a los matemáticos en el proceso de
demostración. Actualmente uno de los mayores avances en este campo ha sido
realizado por la empresa \textit{OpenAI}, que ha conseguido automatizar algunas
demostraciones de olimpíadas de matemáticas~\cite{poluSolvingFormalMath}.


