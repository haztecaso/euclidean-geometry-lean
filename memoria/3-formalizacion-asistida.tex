\section{Formalización asistida por computadores}

En la docencia e investigación en matemáticas es usual escribir las
demostraciones utilizando el lenguaje natural. Se asume que cada paso podría
escribirse en un lenguaje formal, en cuyo caso podría comprobarse que cada paso
deriva de los anteriores y la aplicación de las reglas lógicas. Este trabajo es
demasiado tedioso para los matemáticos, y el exceso de detalles puede restar
protagonismo a las ideas que se quieren exponer.

Sin embargo es una tarea perfecta para ser realizada de forma automática
por máquinas. Los asistentes de demostración nos permiten formalizar
definiciones, enunciados de teoremas, sus demostraciones, y verficar la
corrección de estas pruebas. Algunos de estos sistemas son Coq,
Isabelle/HOL, Agda, Metamath, Mizar y Lean.

Además de un lenguaje, estos sistemas suelen constar de un entorno de desarrollo
que aporta información contextual útil en el proceso de demostración. Lean es
especialmente popular debido a la calidad y facilidad de uso de este entorno
(ver apéndice \ref{sec:entorno}) y a la existencia de \textit{mathlib}, una
amplia librería de matemáticas formalizadas.

Formalizar matemáticas con la ayuda de un asistente de demostraciones consiste
en digitalizar enunciados y resultados escribiéndolos en un lenguaje de
programación que garantiza, mediante una correspondencia entre una teoría de
tipos y la lógica con el lenguaje de la teoría de conjuntos, la validez de cada
paso.

En lo que sigue de esta sección expondremos algunos beneficios y aplicaciones
de los asistentes de demostración. La conferencia \textit{¿Por qué formalizar
	matemáticas?} impartida por María Inés de Frutos en la facultad de
Ciencias Matemáticas de la UCM ha sido la principal referencia a la hora de
elaborar esta lista~\cite{defrutosPorQueFormalizar2023a}.

\paragraph{Comprobación mecanizada de demostraciones.}

Estos sistemas pueden verificar la corrección de demostraciones largas y
técnicas, reduciéndolas a las aplicaciones fundamentales de las reglas del
sistema formal utilizado, como una teoría de tipos.

Un ejemplo en el que se ha conseguido formalizar un resultado muy avanzado es el
del \textit{Liquid Tensor Experiment}. En el año 2020 el medallista Fields
Peter Scholze lanzó a la comunidad de formalización matemática el reto
de verificar un importante teorema publicado por él y Dustin
Clausen~\cite{scholzeLiquidTensorExperiment2022}. Scholze explica que, antes de
que se formalizara el resultado, no estaba totalmente seguro de la corrección de
una parte técnica de la demostración, que contenía muchos detalles técnicos que
la comunidad no estaba estudiando.

La guía de esta demostración, que se terminó de formalizar en el año 2022, se
encuentra en formato web~\cite{scholzeBlueprintLiquidTensor} y es un buen
ejemplo de un documento que presenta resultados paralelamente en un formato
textual, legible por matemáticos, y su formalización en el lenguaje
Lean.

Una de las visiones a futuro de los impulsores de Lean, como Kevin Buzzard, es
la idea de a cada artículo matemático se le asocie su formalización y por tanto
el papel de los revisores pase a enfocarse en el interés de los resultados,
puesto que su corrección estaría garantizada por el asistente de demostración.
Actualmente este objetivo está aún lejos de ser alcanzado.

\paragraph{Digitalización de definiciones y enunciados.}

La creación de una base de datos digital puede mejorar la capacidad de búsqueda
de resultados matemáticos previos, convirtiéndose en un gran apoyo en el proceso
de investigación. El proyecto \textit{Formal
	Abstracts}~\cite{halesFormalAbstracts} se propone realizar una enciclopedia
matemática digital utilizando el lenguaje Lean, con el objetivo de formalizar la
mayor cantidad posible de teoremas publicados en un formato entendible a la vez
por humanos y máquinas. Además se quieren enlazar los términos utilizados en
dichos enunciados con sus definiciones precisas.

El proyecto se encuentra en fase de diseño y queda bastante trabajo para que
alcance sus objetivos, pero muestra una dirección de trabajo muy interesante.

\paragraph{Uso en docencia.}

En ciertos centros universitarios ya se imparten cursos apoyándose en el uso del
asistente Lean. El principal impulsor del uso de estas herramientas en la
docencia es Kevin Buzzard, que desde el año 2021 imparte el curso
\textit{Formalising mathematics}~\cite{buzzardFormalisingMathematicsFormalising}
en el Imperial College.

El uso de asistentes de demostración en la docencia fija una referencia objetiva
sobre qué es una demostración correcta de un resultado, obligando a los
estudiantes a ser precisos en sus razonamientos. Además los estudiantes pueden
aprovechar el entorno interactivo para comprobar autónomamente si sus
razonamientos son correctos, sin necesidad de esperar la corrección del
profesor.

El principal problema actual en este ámbito es que la barrera de entrada al uso
de estos asistentes es alta, puesto que la instalación de las herramientas y
librerías no es simple y es necesaria cierta familiaridad con el lenguaje para
escribir código.

\paragraph{Demostración automatizada.} Se espera que en un futuro se puedan
incorporar a estos sistemas téncicas de inteligencia artificial que proporcionen
indicaciones y ayuda a los matemáticos en el proceso de demostración. Un ejemplo
de avance en este campo es el artículo~\cite{poluSolvingFormalMath} publicado recientemente por por la
empresa OpenAI, en el que se han conseguido automatizar algunas demostraciones de
olimpíadas de matemáticas.


